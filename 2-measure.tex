\newpage
\section{Measure and integration}
\subsection{Products of \( \sigma \)-algebras}
\subsubsection{Defintions of product \sal}
If \( (S_\alpha)_{\alpha \in A} \) are measurable spaces
with \sal s \( \cal F_\alpha \) and
\( S = \prod_\alpha S_\alpha \), then \( S \) has the \emph{product \sal}
\( \cal F = \prod_\alpha \cal F_\alpha = \bigotimes_\alpha \cal F_\alpha \) defined as
\begin{equation}
	\cal F  = \sigma \ob{ \pi_\alpha^{-1}(E_\alpha) \st \alpha \in A, \ E_\alpha \in \cal F_\alpha},
\end{equation}
i.e.\ \emph{generated by one-dimensional measurable cylinders}.
\asteriskline

If \( A \) is countable, then \( \cal F \) is \emph{generated by cuboids} \( \prod_\alpha E_\alpha \), \( \alpha \in A \) (see~\cite[p.\ 23]{fo}). To prove it, define
\begin{equation}
	\cal F_1 = \sigma \ob{\pi_\alpha^{-1}(E_\alpha) \st \alpha \in A, \ E_\alpha \in \cal F_\alpha}, \quad
	\cal F_2 = \sigma \ob{\prod_{\alpha \in A} E_\alpha \st E_\alpha \in \cal F_\alpha}.
\end{equation}
Firstly, \( \pi_\alpha^{-1}(E_\alpha) = \prod_\alpha E_\alpha \) where \( E_\beta = S_\beta \) for \( \beta \neq \alpha \), so \( \cal F_1 \pd \cal F_2 \).

Secondly,
\begin{equation}
	\prod_{\alpha \in A} E_\alpha = \bigcap_{\alpha \in A} \pi_\alpha^{-1}(E_\alpha) \in \cal F_1,
\end{equation}
so that \( \cal F_2 \pd \cal F_1 \).

\asteriskline

Instead of taking \( E_\alpha \)'s from the \sal s, we can restrict ourselves to \emph{sets of generators}. Suppose \( \sigma(\cal E_\alpha) = \cal F_\alpha \), then (the second part again if \( A \) is countable, and proven the same way)
\begin{equation}
	\bigotimes_{\alpha \in A} \cal F_\alpha = \sigma
	\ob{\pi_\alpha^{-1}(E_\alpha) \st \alpha \in A, \ E_\alpha \in \cal E_\alpha} =
	\sigma \ob{\prod_{\alpha \in A} E_\alpha \st E_\alpha \in \cal E_\alpha}.
\end{equation}

To prove the first part, consider the sets
\begin{equation}
	\cal D = \vit{\pi_\alpha^{-1}(E_\alpha) \st \alpha \in A, \ E_\alpha \in \cal E_\alpha}
	,\quad
	\cal C_\alpha = \vit{
		E \in F_\alpha \st \pi_\alpha^{-1} (E) \in \sigma(\cal D)
	}.
\end{equation}
Then \( C_\alpha \) is a \sal \ containing \( \cal E_\alpha \), which means that
\( C_\alpha = \cal F_\alpha \) and \( \sigma(\cal D) = \bigotimes_\alpha \cal F_\alpha \).

\subsubsection{Borel \sal \ on product metric space}
Suppose \( S_j \) are separable metric spaces (\( j \in \N \)) with Borel \sal s
\( \cal B_j \). Then (see~\cite[p.\ 11]{kalle})
\begin{equation}\label{eq:borelprod}
	\cal B(S_1 \times S_2 \times \cdots) = \cal B_1 \otimes \cal B_2 \otimes \cdots
\end{equation}
Consider the sets
\begin{equation}
	\cal C_1 = \vit{A \st A \text{ open in } \bigtimes_j S_j}, \quad
	\cal C_2 = \vit{S_1 \times \cdots \times B_j \times S_{j+1} \times \cdots \st B_j \in \cal B_j},
\end{equation}
and \( \cal C \) defined similar to \( \cal C_2 \) but where \( B_j \) has to be open instead.
The claim is then \( \sigma(\cal C_1) = \sigma(\cal C_2) \).

Clearly \( \cal C \pd  \cal C_1, \cal C_2 \). But it is also clear that \( \sigma(\cal C) \nd \sigma(\cal C_2) \), so
\( \sigma(\cal C) = \sigma(\cal C_2) \). This proves that the \( \nd \) part in~\eqref{eq:borelprod} holds
without separability.

If \( S_j \) are separable, \( \cal C \) is a topological basis of \( \bigtimes_j S_j \), so that
\( \sigma(\cal C) \nd \sigma(\cal C_1) \). Then \( \sigma(\cal C) = \sigma(\cal C_1) \) as well.



\subsubsection{Measurability in coordinate functions}
See~\cite[p.\ 15]{kalle}. Suppose that \( (\Omega, \cal A) \) and \( (S_j, \cal S_j) \) are measurable
spaces for \( j \in \N \), denoting \( S = \bigtimes_j S_j \). Let \( f \colon \Omega \to S \) be a function and define
\( f_j \) as its \emph{coordinate functions} \( f_j = \pi_j \circ f \). Then
\begin{quote}
	\( f \) is measurable if and only if all \( f_j \) are measurable.
\end{quote}

The \enquote{only if} part is trivial, because each \( f_j \) is a composition of measurable \( f \) and \( \pi_j \).
For the \enquote{if} part, note that \( f \) satisfies the definition of measurability on a generating subset, those being
measurable cuboids:
\begin{equation}\label{eq:coordfunc}
	f^{-1}\ob{B} = \bigcap_j f_j^{-1}(B_j), \quad B = B_1 \times B_2 \times \cdots,
\end{equation}
where \( B \) and all \( B_j \) are measurable sets. \bigskip

\emph{Note} that~\eqref{eq:coordfunc} indeed
holds specifically for cuboid sets \( B \) and not generally. Instead of these cuboids, we could have also
chosen the simpler generating set of one-dimensional cylinders \( \pi_j^{-1}(B_j) \).
Note that in all cases pointwise convergence can be replaced with a.e.\ convergence,
which is weaker but without affecting integration.

\subsection{Convergence theorems}
We consider whether \( \int f_n \to \int f \) if \( f \to f_n \) pointwise, where the functions are defined on a \( \sigma \)-finite measure space \( X \). Following~\cite[\textsection 1.4]{du}, we present
the convergence theorem in order: bounded, Fatou's lemma, monotone, dominated.

\subsubsection{Bounded convergence theorem}
Suppose the \( f_n \) have
\begin{itemize}
	\item \emph{bounded domain}: \( f_n \) vanishes on \( E^c \) for some \( E \) with \( \mu(E)<\infty \),
	\item \emph{bounded range}: \( \abs{f_n} \le M \) uniformly,
	\item \( f_n \to f \) \emph{in measure}.
\end{itemize}
Then \( \int f_n \to \int f \).

To \emph{prove} it, we have:
\begin{align}
	\abs{\int f - f_n} & \le \int \abs{f-f_n}                                             \\
	                   & = \int\limits_{\vit{\abs{f-f_n} \ge \varepsilon}} \abs{f-f_n}
	+\int\limits_{\vit{\abs{f-f_n}<\varepsilon}} \abs{f-f_n}                              \\
	                   & \le 2M \mu \ob{\abs{f-f_n} \ge \varepsilon} + \varepsilon \mu(E)
	\to 0, \quad n \ub, \ \varepsilon \downarrow 0.
\end{align}

We implicitly used that \( \abs f \le M \) as well. We obtain this because \( \abs f \ge M + \varepsilon \)
implies \( \abs {f-f_n} \ge \varepsilon \), giving us \( \mu\ob{\abs f > M} = 0 \) after some simple work.

\subsubsection{Fatou's lemma}
Suppose \( f_n \ge 0 \), then
\begin{equation}
	\liminf_{n \ub} \int f_n \ge \int \liminf_{n \ub} f_n.
\end{equation}

\emph{Recall} that
\begin{equation}
	\liminf_{n\ub} f_n = \sup_m \inf_{n\ge m} f_n,
\end{equation}
thus we define \( g_n = \inf_{m \ge n} f_m  \) so that
\( g_n \uparrow g \coloneq \liminf_{n} f_n \).

It then suffices to show (note \( f_n \ge g_n \))
\begin{equation}
	\liminf_{n\ub} \int g_n \ge \int g.
\end{equation}

Let \( E_m \uparrow X \) be measurable sets with \( \mu(E_m) < \infty \). For fixed \( m \):
\begin{equation}
	(g_n \wedge m) \cdot 1_{E_m} \rightarrow (g \wedge m) \cdot 1_{E_m}.
\end{equation}
Finally,
\begin{equation}
	\liminf_{n\ub} \int g_n \ge
	\int_{E_m} g_n \wedge m \rightarrow
	\int_{E_m} g \wedge m \rightarrow \int g.
\end{equation}
The inequality holds on a subsequence, the first convergence (\( n \ub \)) is the bounded
convergence theorem, and the second (\( m \ub \)) is the below lemma.

\asteriskline

In the proof we used the following \emph{lemma} (see~\cite[p.\ 22]{du}): let \( E_n \uparrow X \)
be measurable sets with \( \mu(E_n) < \infty \). Then
\begin{equation}
	\int_{E_n} f \wedge n \rightarrow \int f.
\end{equation}

Clearly the left side is increasing and below \( \int f \). By definition of the
Lebesgue integral, it suffices to show that for every simple nonnegative \( \varphi \),
there exists \( n \) with
\begin{equation}
	\int \varphi \le \int _{E_n} f \wedge n \le \int f.
\end{equation}
Let \( N \) be such that \( \mu(\varphi > N) = 0 \)
and choose \( n \ge N \). Then
\begin{equation}
	\begin{aligned}
		\int_{E_n} f \wedge n & \ge
		\int_{E_n} \varphi = \int \varphi - \int_{E_n ^c} \varphi \\
		                      & \ge \int \varphi - N\mu(E_n^c),
	\end{aligned}
\end{equation}
so that
\begin{equation}
	\liminf_{n \ub} \int_{E_n} f \wedge n \ge \int \varphi,
\end{equation}
since \( \mu(E_n^c) \un \).

\subsubsection{Monotone convergence theorem}
Suppose \( f_n \ge 0 \) and \( f_n \uparrow f \), then
\begin{equation}\label{eq:monconv}
	\int f_n \rightarrow \int f.
\end{equation}

Since \( f_n \le f \) we have \( \limsup_n \int f_n \le \int f \), and
by Fatou's lemma we have \( \liminf_n \int f_n \ge \int f \).

\subsubsection{Dominated convergence theorem}
Let \( f_n \) be such that \( \abs{f_n} \le g \) uniformly,
for some integrable \( g \). If \( f_n \rightarrow f \), then
\begin{equation}
	\int f_n \rightarrow \int f.
\end{equation}

The condition \( \abs{f_n} \le g \) can be written as \( g \pm f_n \ge 0 \). With \( g \pm f_n \to f \),
we apply Fatou's lemma twice. The \( + \) and \( - \) parts respectively give
\( \int f \ge \limsup_n \int f_n \) and
\( \int f \le \liminf_n \int f_n \).


\subsection{Measure extension}

\subsection{Product measure}

\subsection{Fubini's theorem}

\subsection{Notable exercises}

%%%%%%%%%%%%%%

