\newpage
\section{Measure and integration}
\subsection{Products of \( \sigma \)-algebras}
\subsubsection{Defintions of product \sal}
If \( (S_\alpha)_{\alpha \in A} \) are measurable spaces
with \sal s \( \cal F_\alpha \) and
\( S = \prod_\alpha S_\alpha \), then \( S \) has the \emph{product \sal}
\( \cal F = \prod_\alpha \cal F_\alpha = \bigotimes_\alpha \cal F_\alpha \) defined as
\begin{equation}
	\cal F  = \sigma \ob{ \pi_\alpha^{-1}(E_\alpha) \st \alpha \in A, \ E_\alpha \in \cal F_\alpha},
\end{equation}
i.e.\ \emph{generated by one-dimensional measurable cylinders}.
\asteriskline

If \( A \) is countable, then \( \cal F \) is \emph{generated by cuboids} \( \prod_\alpha E_\alpha \), \( \alpha \in A \) (see~\cite[p.\ 23]{fo}). To prove it, define
\begin{equation}
	\cal F_1 = \sigma \ob{\pi_\alpha^{-1}(E_\alpha) \st \alpha \in A, \ E_\alpha \in \cal F_\alpha}, \quad
	\cal F_2 = \sigma \ob{\prod_{\alpha \in A} E_\alpha \st E_\alpha \in \cal F_\alpha}.
\end{equation}
Firstly, \( \pi_\alpha^{-1}(E_\alpha) = \prod_\alpha E_\alpha \) where \( E_\beta = S_\beta \) for \( \beta \neq \alpha \), so \( \cal F_1 \pd \cal F_2 \).

Secondly,
\begin{equation}
	\prod_{\alpha \in A} E_\alpha = \bigcap_{\alpha \in A} \pi_\alpha^{-1}(E_\alpha) \in \cal F_1,
\end{equation}
so that \( \cal F_2 \pd \cal F_1 \).

\asteriskline

Instead of taking \( E_\alpha \)'s from the \sal s, we can restrict ourselves to \emph{sets of generators}. Suppose \( \sigma(\cal E_\alpha) = \cal F_\alpha \), then (the second part again if \( A \) is countable, and proven the same way)
\begin{equation}
	\bigotimes_{\alpha \in A} \cal F_\alpha = \sigma
	\ob{\pi_\alpha^{-1}(E_\alpha) \st \alpha \in A, \ E_\alpha \in \cal E_\alpha} =
	\sigma \ob{\prod_{\alpha \in A} E_\alpha \st E_\alpha \in \cal E_\alpha}.
\end{equation}

To prove the first part, consider the sets
\begin{equation}
	\cal D = \vit{\pi_\alpha^{-1}(E_\alpha) \st \alpha \in A, \ E_\alpha \in \cal E_\alpha}
	,\quad
	\cal C_\alpha = \vit{
		E \in F_\alpha \st \pi_\alpha^{-1} (E) \in \sigma(\cal D)
	}.
\end{equation}
Then \( C_\alpha \) is a \sal \ containing \( \cal E_\alpha \), which means that
\( C_\alpha = \cal F_\alpha \) and \( \sigma(\cal D) = \bigotimes_\alpha \cal F_\alpha \).

\subsubsection{Borel \sal \ on product metric space}
Suppose \( S_j \) are separable metric spaces (\( j \in \N \)) with Borel \sal s
\( \cal B_j \). Then (see~\cite[p.\ 11]{kalle})
\begin{equation}\label{eq:borelprod}
	\cal B(S_1 \times S_2 \times \cdots) = \cal B_1 \otimes \cal B_2 \otimes \cdots
\end{equation}
Consider the sets
\begin{equation}
	\cal C_1 = \vit{A \st A \text{ open in } \bigtimes_j S_j}, \quad
	\cal C_2 = \vit{S_1 \times \cdots \times B_j \times S_{j+1} \times \cdots \st B_j \in \cal B_j},
\end{equation}
and \( \cal C \) defined similar to \( \cal C_2 \) but where \( B_j \) has to be open instead.
The claim is then \( \sigma(\cal C_1) = \sigma(\cal C_2) \).

Clearly \( \cal C \pd  \cal C_1, \cal C_2 \). But it is also clear that \( \sigma(\cal C) \nd \sigma(\cal C_2) \), so
\( \sigma(\cal C) = \sigma(\cal C_2) \). This proves that the \( \nd \) part in~\eqref{eq:borelprod} holds
without separability.

If \( S_j \) are separable, \( \cal C \) is a topological basis of \( \bigtimes_j S_j \), so that
\( \sigma(\cal C) \nd \sigma(\cal C_1) \). Then \( \sigma(\cal C) = \sigma(\cal C_1) \) as well.



\subsubsection{Measurability in coordinate functions}
See~\cite[p.\ 15]{kalle}. Suppose that \( (\Omega, \cal A) \) and \( (S_j, \cal S_j) \) are measurable
spaces for \( j \in \N \), denoting \( S = \bigtimes_j S_j \). Let \( f \colon \Omega \to S \) be a function and define
\( f_j \) as its \emph{coordinate functions} \( f_j = \pi_j \circ f \). Then
\begin{quote}
	\( f \) is measurable if and only if all \( f_j \) are measurable.
\end{quote}

The \enquote{only if} part is trivial, because each \( f_j \) is a composition of measurable \( f \) and \( \pi_j \).
For the \enquote{if} part, note that \( f \) satisfies the definition of measurability on a generating subset, those being
measurable cuboids:
\begin{equation}\label{eq:coordfunc}
	f^{-1}\ob{B} = \bigcap_j f_j^{-1}(B_j), \quad B = B_1 \times B_2 \times \cdots,
\end{equation}
where \( B \) and all \( B_j \) are measurable sets. \bigskip

\emph{Note} that~\eqref{eq:coordfunc} indeed
holds specifically for cuboid sets \( B \) and not generally. Instead of these cuboids, we could have also
chosen the simpler generating set of one-dimensional cylinders \( \pi_j^{-1}(B_j) \).

\subsection{Convergence theorems}
We consider whether \( \int f_n \to \int f \) if \( f \to f_n \) pointwise, where the functions are defined on a \( \sigma \)-finite measure space \( X \). Following~\cite[\textsection 1.4]{du}, we present
the convergence theorem in order: bounded, Fatou's lemma, monotone, dominated.
Note that in all cases pointwise convergence can be replaced with a.e.\ convergence,
which is weaker but without affecting integration.

\subsubsection{Bounded convergence theorem}
Suppose the \( f_n \) have
\begin{itemize}
	\item \emph{bounded domain}: \( f_n \) vanishes on \( E^c \) for some \( E \) with \( \mu(E)<\infty \),
	\item \emph{bounded range}: \( \abs{f_n} \le M \) uniformly,
	\item \( f_n \to f \) \emph{in measure}.
\end{itemize}
Then \( \int f_n \to \int f \).

To \emph{prove} it, we have:
\begin{align}
	\abs{\int f - f_n} & \le \int \abs{f-f_n}                                             \\
	                   & = \int\limits_{\vit{\abs{f-f_n} \ge \varepsilon}} \abs{f-f_n}
	+\int\limits_{\vit{\abs{f-f_n}<\varepsilon}} \abs{f-f_n}                              \\
	                   & \le 2M \mu \ob{\abs{f-f_n} \ge \varepsilon} + \varepsilon \mu(E)
	\to 0, \quad n \ub, \ \varepsilon \downarrow 0.
\end{align}

We implicitly used that \( \abs f \le M \) as well. We obtain this because \( \abs f \ge M + \varepsilon \)
implies \( \abs {f-f_n} \ge \varepsilon \), giving us \( \mu\ob{\abs f > M} = 0 \) after some simple work.

\subsubsection{Fatou's lemma}
Suppose \( f_n \ge 0 \), then
\begin{equation}
	\liminf_{n \ub} \int f_n \ge \int \liminf_{n \ub} f_n.
\end{equation}

\emph{Recall} that
\begin{equation}
	\liminf_{n\ub} f_n = \sup_m \inf_{n\ge m} f_n,
\end{equation}
thus we define \( g_n = \inf_{m \ge n} f_m  \) so that
\( g_n \uparrow g \coloneq \liminf_{n} f_n \).

It then suffices to show (note \( f_n \ge g_n \))
\begin{equation}
	\liminf_{n\ub} \int g_n \ge \int g.
\end{equation}

Let \( E_m \uparrow X \) be measurable sets with \( \mu(E_m) < \infty \). For fixed \( m \):
\begin{equation}
	(g_n \wedge m) \cdot 1_{E_m} \rightarrow (g \wedge m) \cdot 1_{E_m}.
\end{equation}
Finally,
\begin{equation}
	\liminf_{n\ub} \int g_n \ge
	\int_{E_m} g_n \wedge m \rightarrow
	\int_{E_m} g \wedge m \rightarrow \int g.
\end{equation}
The inequality holds on a subsequence, the first convergence (\( n \ub \)) is the bounded
convergence theorem, and the second (\( m \ub \)) is the below lemma.

\asteriskline

In the proof we used the following \emph{lemma} (see~\cite[p.\ 22]{du}): let \( E_n \uparrow X \)
be measurable sets with \( \mu(E_n) < \infty \). Then
\begin{equation}
	\int_{E_n} f \wedge n \rightarrow \int f.
\end{equation}

Clearly the left side is increasing and below \( \int f \). By definition of the
Lebesgue integral, it suffices to show that for every simple nonnegative \( \varphi \),
there exists \( n \) with
\begin{equation}
	\int \varphi \le \int _{E_n} f \wedge n \le \int f.
\end{equation}
Let \( N \) be such that \( \mu(\varphi > N) = 0 \)
and choose \( n \ge N \). Then
\begin{equation}
	\begin{aligned}
		\int_{E_n} f \wedge n & \ge
		\int_{E_n} \varphi = \int \varphi - \int_{E_n ^c} \varphi \\
		                      & \ge \int \varphi - N\mu(E_n^c),
	\end{aligned}
\end{equation}
so that
\begin{equation}
	\liminf_{n \ub} \int_{E_n} f \wedge n \ge \int \varphi,
\end{equation}
since \( \mu(E_n^c) \un \).

\subsubsection{Monotone convergence theorem}
Suppose \( f_n \ge 0 \) and \( f_n \uparrow f \), then
\begin{equation}\label{eq:monconv}
	\int f_n \rightarrow \int f.
\end{equation}

Since \( f_n \le f \) we have \( \limsup_n \int f_n \le \int f \), and
by Fatou's lemma we have \( \liminf_n \int f_n \ge \int f \).

\subsubsection{Dominated convergence theorem}
Let \( f_n \) be such that \( \abs{f_n} \le g \) uniformly,
for some integrable \( g \). If \( f_n \rightarrow f \), then
\begin{equation}
	\int f_n \rightarrow \int f.
\end{equation}

The condition \( \abs{f_n} \le g \) can be written as \( g \pm f_n \ge 0 \). With \( g \pm f_n \to f \),
we apply Fatou's lemma twice. The \( + \) and \( - \) parts respectively give
\( \int f \ge \limsup_n \int f_n \) and
\( \int f \le \liminf_n \int f_n \).


\subsection{Measure extension}
Carath\' eodory's theorem (multiple variations) extends measure from family of subsets to whole \sal \  (unique?); Kolmogorov's theorem extends measures on finite-dimensional subspaces to a single measure on an infinite-dimensional metric probability space.


\subsection{Product measure}
Suppose \( (S, \cal S, \mu) \) and \( (T, \cal T, \nu) \) are
\( \sigma \)-finite measure spaces. We wish to introduce a measure
on the product measurable space \( (S \times T, \cal S \otimes \cal T) \).

\subsubsection{Sections}
For any \emph{measurable set} \( E \in \cal S \otimes \cal T \), \( s \in S \) and
\( t \in T \) we define the sections
\begin{equation}
	E_s = \vit{t \in T \st (s, t) \in E}, \quad
	E^t = \vit{s \in S \st (s, t) \in E}.
\end{equation}

For any measurable \emph{function} \( f \colon S \times T \to U  \) and
any \( s \in S \) and \( t \in T \) we define the sections
\( f_s \colon T \to U \) and \( f^t \colon S \to U \) with
\begin{equation}
	f_s(t) = f(s,t), \quad f^t(s) = f(s, t).
\end{equation}

We claim that the sections are \emph{always measurable} (sets or functions).
Consider first the set \( E_s \).
Define
\begin{equation}
	\cal D = \vit{E \in \cal S \otimes \cal T \st E_s \in \cal T \text{ for all } s \in S}.
\end{equation}
Clearly, \( \cal D \) is a \( \lambda \)-system and
contains the generating \( \pi \)-system of rectangles
\( A \times B \) since \( (A \times B)_s = B \) if \( s \in A \) and
\( \emptyset \) otherwise. By Dynkin's lemma, \( \cal D = \cal S \otimes \cal T \).

To prove the measurability of functions \( f_s \), it is simple to show that
\( f_s^{-1}(B) = \ob{f^{-1}(B)}_s \in \cal T \), for any measurable \( B \).

\asteriskline
Let \( f \colon S \times T \to \R_{\ge 0} \) be a measurable functions.
We claim that the \emph{maps}
\begin{equation}
	s \mapsto \int_T f(s,t) \ime \nu t, \quad
	t \mapsto \int_S f(s,t) \ime \mu s
\end{equation}
are measurable.

We use a type of Lebesgue's induction.
Assume first that \( \mu \) and \( \nu \) are finite
and that \( f = 1_E \) where \( E = A \times B \).
Then
\begin{equation}
	\int_T 1_E(s,t) \ime \nu t = \nu(E_s) = \nu(B) \cdot  1_A(s).
\end{equation}
Clearly, \( s \mapsto \nu(B)\cdot 1_A(s) \) is measurable. We wish to
extend this to \emph{arbitrary indicator function} so we define
\begin{equation}\label{eq:calDsec3}
	\cal D = \vit{E \in \cal S \otimes \cal T \st s \mapsto \nu(E_s) \text{ is measurable}}
\end{equation}
Again, \( \cal D \) is a \( \lambda \)-system containing
the generating \( \pi \)-system of measurable rectangles, so
\( \cal D = \cal S \otimes \cal T \).

This is then extended to simple functions (linearity of integration)
and to general nonnegative measurable functions. For the latter,
note that monotone convergence of integrals \( \int_T f_n(s,t) \ime \nu t\)
means the pointwise convergence of the maps \( s \mapsto \int_T f_n(s,t) \ime \nu t \),
and the class of measurable functions is closed under
pointwise limits (see~\cite[ex.\ 1.2.7]{du}) \bigskip

\emph{Note.} The requirement \( f \ge 0 \) serves to insure
that the integrals exist. An alternative requirement is
\( \int \abs f \D \xi < \infty \) (with \( \xi \) product measure).

\emph{Note.} The assumption of finite measures (important for \( \cal D \) being a \( \lambda \)-system) does not lose on
generality because of the following: take \( F_n \uparrow T \) with
\( \nu(F_n) < \infty \). Then the maps
\( s \mapsto \nu(E_s \cap F_n) \) converge pointwise to
\( s \mapsto \nu(E_s) \) with pointwise convergence preserving measurability.
The family
\begin{equation}
	\cal D' = \vit{E \in \cal S \otimes \cal T \st
		s \mapsto \nu(E_s \cap F_n) \text{ is measurable for all } n}
\end{equation}
is certainly a \( \lambda \)-system, ensuring the measurability of all
\( s \mapsto \nu(E_s \cap F_n) \).

\subsubsection{Definitions}
We can define a measure \( \xi \) on measurable rectangles with
\begin{equation}\label{eq:defprodm1}
	\xi(A \times B) = \mu(A)\nu(B), \quad A \in \cal S, \ B \in \cal T.
\end{equation}
The function \( \xi \) is then extended to a unique measure on
\( \cal S \otimes \cal T \) by Carath\' eodory-type theorems.

\asteriskline

\emph{Alternatively}, we can define
\begin{equation}\label{eq:defprodm2}
	\xi'(E) = \int_T \mu(E^t) \ime \nu t =
	\int_S \nu(E_s) \ime \mu s,
	\quad E \in \cal S \otimes \cal T.
\end{equation}

\asteriskline

Let us \emph{prove} these definitions are equivalent. Formally using
Fubini's theorem:
\begin{equation}\label{eq:prodmfubini0}
	\begin{aligned}
		\xi(E) & = \int_{S \times T} 1_{E}(u) \ime \xi u           \\
		       & = \int_S \ime \mu s \int_T 1_{E}(s, t) \ime \nu t \\
		       & = \int_S \nu(E_s) \ime \mu s = \xi'(E).
	\end{aligned}
\end{equation}

Conversely, for \( E = A \times B \),
\begin{equation}\label{eq:prodmalt}
	\xi'(E) = \int_S \nu(E_s) \ime \mu s =
	\int_T \nu(B) 1_A(s) \ime \mu s = \mu(A) \nu(B) = \xi(E).
\end{equation}

\asteriskline

We should also \emph{justify} the second equality in~\eqref{eq:defprodm2}.
Define \( \xi'(E) = \int_T \mu(E^t) \ime \nu s \)
and \( \xi''(E) = \int_S \nu(E_s) \ime \mu t \).
By~\eqref{eq:prodmalt} \( \xi' \) and
\( \xi'' \) agree on the generating \( \pi \)-system of
measurable rectangles. By a well known lemma, \( \xi' = \xi '' \).
This argument also suffices to prove \( \xi=\xi' \) (Fubini's theorem is not necessary, \eqref{eq:prodmfubini0} serves a heuristic purpose).

\subsubsection{Fubini's theorem}
Let \( f \colon S \times T \to \R \) be a measurable function such that
\( \int_{S \times T} \abs f \D \xi < \infty \). Then
\begin{equation}\label{eq:prodmfubini}
	\int_{S \times T} f \D \xi
	= \int_S \ime \mu s \int_T f(s, t) \ime \nu t
	= \int_T \ime \nu t \int_S f(s, t) \ime \nu s.
\end{equation}

Consider first a measurable function \( f \ge 0 \).
By previous discussion, the integrals are well-defined and \( \xi \) satisfies:
\begin{equation}\label{eq:prodmfubini2}
	\xi(E) = \int_S \ime \mu s \int_T 1_E(s, t) \ime \nu t, \quad E \in \cal S \otimes \cal T,
\end{equation}
where we also know the order of integration can be reversed. Thus~\eqref{eq:prodmfubini} holds for indicator functions, and is extended to arbitrary
nonnegative functions by linearity and monotone convergence. For general integrable \( f \), apply the previous case to \( f^+ \) and \( f^- \) and subtract. Integrability ensures
that the integrands are measurable and the difference is well-defined.
%\cite[p.\ 25]{kalle} notes that \( f \) can be infinite only on set of measure \( 0 \).; why?

\subsection{Notable exercises}
\subsubsection{Equality of measures agreeing on a generating \( \pi \)-system}
Let \( \mu \) and \( \nu \) be finite measures on measurable space
\( (S, \cal S) \). If
\begin{equation}
	\mu(C) = \nu(C), \quad C \in \cal C,
\end{equation}
where \( \cal C \) is a generating \( \pi \)-system, then \( \mu=\nu \). This can be
extended to \( \sigma \)-finite measures.

To \emph{prove} this, note that
\begin{equation}
	\cal D = \vit{E \in \cal S \st \mu(E) = \nu(E)}
\end{equation}
is a \( \lambda \)-system containing \( \cal C \).

For \emph{\( \sigma \)-finite} measures, note that the previous implies
that \( \mu=\nu \)  when restricted to finite measure spaces
\( (S\cap E_n, \cal S \cap E_n) \), where \( E_n \uparrow S \)
and \( \mu(E_n) < \infty \).
Then for any \( E \in \cal S \)
\begin{equation}
	\mu(E) = \lim_{n \ub} \mu(E \cap E_n) = \lim_{n \ub} \nu(E \cap E_n) = \nu(E).
\end{equation}

\subsubsection{Pointwise convergence preserves measurability}
See~\cite[ex.\ 1.3.7]{du}. Suppose \( f_n \to f \) pointwise with
\( f_n \) measurable. Then \( f \) is measurable because
\begin{equation}
	\vit{f \le a} =
	\bigcup_{r \in \N}
	\bigcup_{k=1}^\infty
	\bigcap_{n \ge k}
	\vit{f_n \le a + \frac 1r}.
\end{equation}

\subsubsection{Tail-formula for \( p \)-th moment}
If \( f \) is a measurable function, then
\begin{equation}
	\norm f_p
	= \ob{p\int_\R y^{p-1} \mu ({\abs {f(x)} \ge y) \D y} }^{1/p}.
\end{equation}

It can be derived, using Fubini for nonnegative functions:
\begin{equation}
	\begin{aligned}
		\norm f_p^p & = \int_\R \abs{f(x)}^p \D x                                    \\
		            & = \int_\R \int_0^{\abs{f(x)}} py^{p-1} \D y \D x               \\
		            & = \int_\R py^{p-1} \int_\R 1_{\cc{0,\abs{f(x)}}}(y)  \D x \D y \\
		            & = p\int_\R y^{p-1} \mu (\abs{f(x)} \ge y) \D y.
	\end{aligned}
\end{equation}

\asteriskline
This holds for all \( 0 < p < \infty \). In probability, this is written
\begin{equation}
	\E(X ^p) = p \int_\R x^{p-1}\P(X \ge x) \D x, \quad X \ge 0.
\end{equation}


\subsubsection{\( \infty \)-norm as limit of \( p \)-norms}
We claim that
\begin{equation}
	\norm f_\infty = \lim_{p \ub} \norm f_p.
\end{equation}
where \( f \) is a measurable function on a probability space.

Let \( \varphi = \sum_{j=1}^n a_j 1_{A_j} \) be a \emph{simple} function
on an \emph{arbitrary} measure space, so that
\( \norm \varphi_p^p = \sum_j \abs {a_j}^p \mu(A_j) \).
Define also \( M = \max \vit{\abs{a_j} \st 1 \le j \le n } = \norm \varphi_\infty  \)
and let \( m \) be the corresponding index.
Clearly \( \norm \varphi_p \le \norm \varphi_\infty \), but also
\begin{equation}
	\norm \varphi_p \ge M \ob{\mu(A_m)}^{1/p} \rightarrow M, \quad p \ub,
\end{equation}
proving the claim for simple functions on arbitrary measure spaces.

Similar ideas are used for general \( f \) on a \emph{probability} space.
Again \( \norm f_p \le \norm f_\infty \). Take arbitrary \( \varepsilon > 0 \)
and set \( A_\varepsilon = \vit{\abs f > M - \varepsilon} \). By definition of
\( M \), \( 0 < \mu(A_\varepsilon) < \infty \). Then,
\begin{equation}
	\norm f_p^p \ge \int_{A_\varepsilon} \abs f^p \D \mu
	\ge (M-\varepsilon)^p \mu(A_\varepsilon).
\end{equation}
Again \( \liminf_p \norm f_p \ge M-\varepsilon \), and the claim follows
by \( \varepsilon \downarrow 0 \).

% \subsubsection{Density of simple functions in \( \L^p \) spaces}

% \subsubsection{Density of step functions in \( \L^p(\R) \) spaces}

%todo:
% luzin, egorov, etc.?



%%%%%%%%%%%%%%

