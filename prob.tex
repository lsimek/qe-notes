\documentclass[12pt]{scrartcl}
% \documentclass[12pt]{article}
% \usepackage[v1]{subfiles} %%

\usepackage[utf8]{inputenc}
% \usepackage[croatian]{babel}
\usepackage{datetime}
\usepackage[a4paper,margin=0.75in]{geometry}
\setlength\baselineskip{15pt}

\usepackage{amsfonts}
\usepackage{amsmath}
\usepackage{amssymb}
\usepackage{amsthm}
\usepackage{csquotes}
\usepackage{tikz}
\usepackage{arydshln}
\pagenumbering{gobble}
\usepackage{float}
\usepackage[dvipsnames]{xcolor}
%\usepackage{mathptmx} % times new roman
% \usepackage{mathpazo} % tex gyre pagella
% \usepackage{baskervaldx}
% \usepackage[baskervaldx]{newtxmath}
% \usepackage[T1]{fontenc}
% \fontseries{ub}\selectfont

\usepackage{breqn}
\usepackage{thmtools}
\usepackage{multirow}
\usepackage{mathtools}
\usepackage[unicode]{hyperref}

\usepackage{booktabs}
\usepackage{listings}

%%%

% Save original macros
% --------------------
\usepackage{letltxmacro}

\LetLtxMacro\OriginalLongrightarrow\Longrightarrow
\LetLtxMacro\OriginalLongleftarrow\Longleftarrow

% Implement new macros
% --------------------
\usepackage{trimclip}
\DeclareRobustCommand\Longrightarrow{\NewRelbar\joinrel\Rightarrow}
\DeclareRobustCommand\Longleftarrow{\Leftarrow\joinrel\NewRelbar}

\makeatletter
\DeclareRobustCommand\NewRelbar{%
  \mathrel{%
    \mathpalette\@NewRelbar{}%
  }%
}
\newcommand*\@NewRelbar[2]{%
  % #1: math style
  % #2: unused
  \sbox0{$#1=$}%
  \sbox2{$#1\Rightarrow\m@th$}%
  \sbox4{$#1\Leftarrow\m@th$}%
  \clipbox{0pt 0pt \dimexpr(\wd2-.6\wd0) 0pt}{\copy2}%
  \kern-.2\wd0 %
  \clipbox{\dimexpr(\wd4-.6\wd0) 0pt 0pt 0pt}{\copy4}%
}
\makeatother
%%%


\definecolor{seagreen}{HTML}{21b2aa}
\definecolor{magenta}{HTML}{b2217f}
\definecolor{gold}{HTML}{ca9520}
\definecolor{red}{HTML}{da272f}
\definecolor{blue}{HTML}{4682b4}
\definecolor{navy}{HTML}{7b000b} %% orig 06038d % onda 7b0000 % onda 7b000b
\definecolor{nred}{HTML}{650015}
\definecolor{ngreen}{HTML}{156515}

%% 
\definecolor{highlightcolor}{named}{ngreen}
\addtokomafont{disposition}{\color{highlightcolor}}
\renewcommand{\emph}[1]{\textbf{\textcolor{highlightcolor}{#1}}}
%%

\definecolor{codebg}{HTML}{f0f0f0}
\definecolor{commentfg}{HTML}{333333}

\lstdefinestyle{mystyle}{
	backgroundcolor=\color{codebg},
	commentstyle=\color{commentfg},
	keywordstyle=\color{navy},
	stringstyle=\color{commentfg},
	basicstyle=\ttfamily\footnotesize,
	breakatwhitespace=false,
	breaklines=true,
	captionpos=b,
	keepspaces=true,
	numbers=none,
	showspaces=false,
	showstringspaces=false,
	showtabs=false,
	tabsize=2
}
\lstset{style=mystyle}

\hypersetup{
    colorlinks,
    linkcolor=highlightcolor,
    citecolor=highlightcolor,
    urlcolor=highlightcolor
}

\usepackage{enumitem}
\makeatletter
\providecommand*{\blx@noerroretextools}{}
\makeatother

\usepackage[
%	style=alphabetic,
	style=numeric,
	giveninits
	% firstinits=true,
	% terseinits=true,
	% giveninits=true
]{biblatex}
\addbibresource{literatura.bib}
\let\oldcite\cite
\renewcommand{\cite}[2][]{%
    \def\tmp{#1}%
    \ifx\tmp\empty
        \mbox{\oldcite{#2}}%
    \else
        \mbox{\oldcite[\tmp]{#2}}%
    \fi
}
\usepackage{pgfplots}
\pgfplotsset{compat=1.15}
\usepackage{mathrsfs}
\usetikzlibrary{arrows}
	
\usepackage[section]{placeins}
\declaretheorem{teorem}
\numberwithin{teorem}{subsection}
\numberwithin{equation}{section}
\numberwithin{figure}{section}
\numberwithin{table}{section}
\declaretheorem[sibling=teorem, style=plain]{lemma}
\declaretheorem[style=remark, sibling=teorem]{note}
%\declaretheorem[style=remark, sibling=teorem]{komentar*}
\declaretheorem[style=definition, sibling=teorem]{definition}
\declaretheorem[sibling=teorem]{corollary}
% \declaretheorem[style=remark, sibling=teorem]{zadatak}
\declaretheorem[sibling=teorem]{proposition}
\declaretheorem[sibling=teorem]{example}
\declaretheorem[sibling=teorem]{problem}
\declaretheorem[sibling=teorem]{exercise}

\newcommand{\ub}{\rightarrow \infty}
\newcommand{\un}{\rightarrow 0}
\newcommand{\norm}[1]{\left\lVert#1\right\rVert} 
% \newcommand{\given}{\, \middle\vert \,}
\newcommand{\given}{\, \vert \,}
\newcommand{\st}{\, \colon \,}
\newcommand{\D}{\,\mathrm d}
\newcommand{\ani}{{^\perp}}
\newcommand{\ran}{\mathrm{ran}\,}
\newcommand{\bop}{\mathrm{B}}
\newcommand{\skp}[2]{\left\langle {#1}, {#2}  \right\rangle}
\newcommand{\jpod}{\stackrel{\mathrm d}{=}}
\newcommand{\jgs}{\stackrel{\mathrm{g.s.}}{=\joinrel=}}
%\newcommand{\jzbog}[1]{\stackrel{\eqref{#1}}{=}}
\newcommand{\stackrell}[2]{\stackrel{\scriptstyle {#1}}{#2}}
\newcommand{\konvp}{\stackrel{\mathbb P}{\longrightarrow}}
\newcommand{\konvw}{\stackrel{\mathrm w}{\longrightarrow}}
\newcommand{\konvd}{\stackrel{\mathrm d}{\longrightarrow}}
\newcommand{\konvgs}{\stackrel{\mathrm{g.s.}}{\longrightarrow}}
\DeclareMathOperator{\card}{card}
\DeclareMathOperator{\var}{Var}
\DeclareMathOperator{\supp}{supp}
\DeclareMathOperator{\im}{im}
\DeclareMathOperator{\olspan}{\overline{span}}
\DeclareMathOperator{\tg}{tg}
\DeclareMathOperator{\ctg}{ctg}
\DeclareMathOperator{\arctg}{arctg}
\DeclareMathOperator{\Arsh}{Arsh}
% \DeclareMathOperator{\ln}{\ln}

\newcommand{\ol}[1]{\overline{#1}}
\newcommand{\wh}[1]{\widehat{#1}}
\newcommand{\wt}[1]{\widetilde{#1}}
\renewcommand{\th}[1]{\hat{#1}} % drukciji hat za tekst?
% \newcommand{\eqdef}{\ {=\colon} \ } 
% \newcommand{\defeq}{\ {\colon\!=} \ }

\newcommand{\R}{\mathbb{R}}
\newcommand{\N}{\mathbb{N}}
\newcommand{\Z}{\mathbb{Z}}
\newcommand{\Q}{\mathbb{Q}}
\newcommand{\C}{\mathbb{C}}
\renewcommand{\P}{\mathbb{P}}
\newcommand{\Ptrans}{\P}  % trazicijska funkcija Markovljevog procesa
\newcommand{\E}{\mathbb{E}}
\def\B{\mathrm{B}}
\def\K{\mathrm{K}}
\def\M{\mathrm{M}}
\def\CC{\mathrm{C}}
\def\RR{\mathrm{R}}
\def\H{\mathcal{H}}
\def\K{\mathcal{K}}
% \def\Z{\mathcal{Z}}
\def\S{\mathcal{S}}
\def\T{\mathcal{T}}
\def\U{\mathcal{U}}

\newcommand{\cz}{Calder\' on--Zygmund}
\newcommand{\holder}{H\" older}
\newcommand{\cadlag}{c\` adl\` ag}
\newcommand{\levy}{L\' evy}
\newcommand{\ito}{It\= o}
\newcommand{\dnorm}{\norm{\, \cdot \,}}
\def\L{\mathrm{L}}
\def\w{\mathrm{w}}

%%
\newcommand{\oo}[1]{\left\langle {#1} \right\rangle}
\newcommand{\oc}[1]{\left\langle {#1} \right]}
\newcommand{\co}[1]{\left[ {#1} \right\rangle}
\newcommand{\cc}[1]{\left[ {#1} \right]}

\newcommand{\ob}[1]{\left( {#1} \right)}
\newcommand{\ug}[1]{\left[ {#1} \right]}
\newcommand{\vit}[1]{\left\{ {#1} \right\}}

\newcommand{\pr}[3]{\bigcap\limits_{#1}^{#2} {#3}}
\newcommand{\uni}[3]{\bigcup\limits_{#1}^{#2} {#3}}
\newcommand{\suma}[3]{\sum\limits_{#1}^{#2} {#3}}

% inline
\newcommand{\ipr}[3]{\cap_{#1}^{#2} {#3}}
\newcommand{\iuni}[3]{\cup_{#1}^{#2} {#3}}
\newcommand{\isuma}[3]{\sum_{#1}^{#2} {#3}}

\newcommand{\up}[1]{\mathrm{#1}}
\renewcommand{\cal}[1]{\mathcal{#1}}

\newcommand{\pd}{\subseteq}
\newcommand{\nd}{\supseteq}

\newcommand{\rz}{\setminus}

\renewcommand{\le}{\leqslant}
\renewcommand{\ge}{\geqslant}

\newcommand{\ime}[2]{{#1}({\mathrm d {#2}})}

\newcommand{\io}{\ \text{i.o.}}
\newcommand{\ult}{\ \text{ult.}}

\newcommand{\whi}{\, ; \,}
\newcommand{\zerone}{\( 0 \)-\( 1 \)}
%%

%% specificno za fajl
\newcommand{\sal}{\( \sigma \)-algebra}
%%


%\newcommand{\distr}[3]{\left\{ x \in {#1} \st \left| {#2}(x) \right| \ge {#3} \right\}}
\newcommand{\abs}[1]{\left| {#1} \right|}
% \newcommand{\fave}[2]{\left\langle {#1} \right\rangle_{#2}}
\let\etoolboxforlistloop\forlistloop % save the good meaning of \forlistloop
\usepackage{autonum}
\let\forlistloop\etoolboxforlistloop % restore the good meaning of \forlistloopi

\newcommand{\asteriskline}{
\vskip20pt
\centerline{
\Large{
\textasteriskcentered\hspace{.1em}
\raisebox{.87ex}\textasteriskcentered\hspace{.1em}
\textasteriskcentered
}}
\vskip20pt
}

\newcommand{\astfootnote}[1]{%
\let\oldthefootnote=\thefootnote%
\setcounter{footnote}{0}%
\renewcommand{\thefootnote}{\fnsymbol{footnote}}%
\footnote{#1}%
\let\thefootnote=\oldthefootnote%
}

\begin{document}
\pagenumbering{gobble}
\title{Probability Theory}
\subtitle{Cheat Sheet \& Study materials\astfootnote{Luka Šimek, 2025./2026.}}
\date{}

%\date{\today}
\maketitle
\vspace{-5em}
\begin{flushleft}
	These materials...
\end{flushleft}
\vspace{-.5em}


\vspace{.25em}  % Add some space before signature???
\tableofcontents
\newpage
\pagenumbering{arabic}

\section{Set structures}
\subsection{Definitions}
Let \( S \) be a set and \(  \cal S \pd \cal P \ob S \) a family of subsets. We define a number of terms: \( \cal S \) is a
\begin{enumerate}[label=(\alph*)]
	\item \emph{semiring} if
	      \begin{itemize}
		      \item \( A, B \in \cal S \implies A \cap B \in \cal S \)
		      \item if \(A, B \in \cal S  \) then \( A \setminus B \) is a finite disjoint union of sets in \( \cal S \)
	      \end{itemize}

	\item \emph{ring} if
	      \begin{itemize}
		      \item \( A, B \in \cal S \implies A \cap B \in \cal S \)
		      \item \( A, B \in \cal S \implies A \rz B \in \cal S \)
	      \end{itemize}

	\item \emph{semialgebra} if
	      \begin{itemize}
		      \item \( A, B \in \cal S \implies A \cap B \in \cal S \)
		      \item \( A \in \cal S \implies A^c \) is a finite disjoint union of sets in \( \cal S \)
	      \end{itemize}

	\item \emph{algebra} if it is closed under complements and \emph{finite} unions.

	\item \emph{monotone class} if \( A_j \uparrow A \) and \( A_j \downarrow A \) imply \( A \in \cal S \)

	\item \emph{$\sigma$-algebra} if...

	\item \emph{$\pi$-system} if \( A, B \in \cal S \implies A \cap B \in \mathcal S \).

	\item \emph{$\lambda$-system} if
	      \begin{itemize}
		      \item $S \in \cal S$
		      \item \( A \pd B, \  A,B \in \cal S \implies B \rz A \in \cal S \)
		      \item \( A_j \uparrow A \implies A \in \cal S \)
	      \end{itemize}
\end{enumerate}

\asteriskline

Clearly, a \( \sigma \)-algebra is all of the above. Also:\begin{itemize}
	\item if \( \cal S \) is both a monotone class and an algebra, it is a \( \sigma \)-algebra,
	\item if \( \cal S \) is an algebra closed with respect to disjoint unions, it is a \sal,
	\item if \( \cal S \) is both a \( \pi \)-system and \( \lambda \)-system, it is a \( \sigma \)-algebra.
\end{itemize}

\subsection{Equivalent definition of \( \lambda \)-system}
A \( \lambda \)-system can also be defined by:
\begin{itemize}
	\item \( S \in \cal S \)
	\item \( A \in \cal S \implies A^c \in \cal S \)
	\item if \( A_j \in \cal S \) with \( j \in \N \) are disjoint, then \( \iuni j{}A_j \in \cal S \).
\end{itemize}

Suppose the first definition and let us prove the other. We simply get complements from \( A^c = S \rz A \) and \( S \in \cal S \). Suppose \( A_1 \) and \( A_2 \) are disjoint. Then \( A_1 \pd A_2^c \) so \( A_2^c \rz A_1 \in \cal S \). This means we have finite disjoint unions because
\begin{equation}
	A_2^c \rz A_1 = A_2^c \cap A_1^c = \ob{A_1 \cup A_2}^c.
\end{equation}
With finite disjoint unions,
\begin{equation}
	\uni{j=1}\infty{A_j} = \uni{j=1}\infty{\ug{\uni{k=1}j{A_j}}} \in \cal S
\end{equation}
as this is a monotone union.

Now let us prove the other direction. If \( A \pd B \) then \( B \rz A = B \cap A^c = \ob{B^c \cup A }^c \). Then suppose \( A_j \) are monotone increasing. We have the disjoint union
\begin{equation}
	\uni{j=1}\infty{A_j} = A_1 \cup \uni{j=2}\infty{\ug{A_j \setminus \uni{k=1}{j-1}A_k}} \in \cal S
\end{equation}
as all the set differences are proper.

\subsection{Dynkin--Sierpinski \( \pi \)-\( \lambda \) lemma}
Let \( S \) be a set, \( \cal C \) a \( \pi \)-system and \( \cal D \) a \( \lambda \)-system. Then
\begin{equation}
	\cal C \pd \cal D \implies \sigma \ob {\cal C} \pd \cal D.
\end{equation}

The proof is from~\cite[p.\ 10]{kalle}. Clearly we can assume that \( \cal D \) is the
smallest possible \( \lambda \)-system, this being denoted \( \lambda(\cal D) \). Taking into account that \( \pi \) and \( \lambda \) together make a \( \sigma \)-algebra, the statement becomes:
\begin{quote}
	If \( \cal C \) is a \( \pi \)-system, then \( \lambda (\cal C) \) is also a \( \pi \)-system.
\end{quote}

Therefore, we need to prove \( A \cap B \in \cal D \) whenever \( A, B \in \cal D \).
For an arbitrary \( B \in \cal C \) define
\begin{equation}
	\cal S_B = \vit{A \in \cal D \st A \cap B \in \cal D}.
\end{equation}
Clearly \( \cal S_B \) is a \( \lambda \)-system containing \( \cal C \) (because \( \cal C \) is itself a \( \pi \)-system). Since \( \cal D = \lambda(\cal C) \), it follows
that \( \cal D \pd \cal S_B \). That is, \( A \cap B \in \cal D \) whenever
\( B \in \cal C \) and \( A \in \cal D \).

Now for any \( A \in \cal D \) define
\begin{equation}
	\cal S_A' = \vit{B \in \cal D \st A \cap B \in \cal D}.
\end{equation}
Similarly, and also using the previous result, we get \( \cal D \pd \cal S_A' \)
and we conclude that \( A \cap B \in \cal D \) whenever \( A, B \in \cal D \).

\asteriskline
Suppose \( \cal C \pd \cal D \pd \cal F \) where \( \cal F \) is a \( \sigma \)-algebra on
a probability space. Let \( \cal D \) be the set of all events with a certain property, this property naturally making it a \( \lambda \)-system. Suppose also \( \sigma(\cal C) = \cal F \).
By the lemma, \( \cal D = \cal F \), that is, all events have the desired property.

\subsection{Notable exercises}
\subsubsection{Any open set of reals is a countable union of intervals} See~\cite[ex.\ 1.1.2]{du}. Define
\begin{equation}\label{eq:calsd}
	\cal S_d = \vit{\prod_{j=1}^d \oc{a_j, b_j} \st -\infty \le a < b \le +\infty} \pd \R^d.
\end{equation}
Note that \( \cal S_d \) is a semialgebra (automatically, also a \( \pi \)-system). We claim that
\( \sigma(\cal S_d) = \cal B(\R^d) \). It suffices to prove that every open set in \( \R^d \) can
be written as the countable union of sets in \( \cal S_d \).

Take \( d=1 \) with a simple extension afterwards. Let \( A \pd \R \) be open, so that for each \( x \in A \) there is \( \varepsilon_x \) such that \( B(x,\varepsilon_x) \pd A \). Also,
\begin{equation}
	A = \uni{x \in A}{}{B(x,\varepsilon_x)}.
\end{equation}
Note that \( A \cap \Q \) is dense in \( A \). Namely, each \( x \) is within its ball, and within the ball is a sequence of rationals converging to \( x \).
For each rational \( y \) such that \( \abs{x-y} < \varepsilon_x/2 \) we can clearly choose an \( \varepsilon_y \) such that
\( x \in B(y,\varepsilon_y) \pd A \). This gives us the countable union\footnote{while different \( x \) can give different \( \varepsilon_y \), these clearly have a finite supremum}
\begin{equation}
	A = \uni{y \in A \cap \Q}{}{B(y,\varepsilon_y)}
\end{equation}


\asteriskline

\subsubsection{Infinite unions of algebras/\( \sigma \)-algebras} See~\cite[ex.\ 1.1.4]{du}.
Suppose \( \cal F_1 \pd \cal F_2 \pd \cdots  \) and \( \cal F = \cup_j \cal F_j \).

If \( \cal F_j \) are algebras, then \( \cal F \) is also an algebra because
\begin{equation}
	\uni{j\in N}{}{A_j} \in \cal F_{\max N}, \quad A_j \in \cal F_j.
\end{equation}
where \( N \pd \N \) is finite.

The analogous claim is not true if \( \cal F_j \) are \( \sigma \)-algebras.
Let \( S = \R \) and
\begin{equation}
	\cal F_j = \sigma \ob{\co{\frac n{2^j}, \frac{n+1}{2^j}} \st n \in \N}, \quad j \in \N.
\end{equation}
If \( \cal F \) is a \( \sigma \)-algebra,
\begin{equation}
	\vit 0 = \pr{j \in \N}{}{\co{0,\frac1{2^j}}} \in \cal F,
\end{equation}
but there is no \( j \) such that \( \vit 0 \in \cal F_j \),
giving us a contradiction.

\newpage
\section{Measure and integration}
\subsection{Products of \( \sigma \)-algebras}
\subsubsection{Defintions of product \sal}
If \( (S_\alpha)_{\alpha \in A} \) are measurable spaces
with \sal s \( \cal F_\alpha \) and
\( S = \prod_\alpha S_\alpha \), then \( S \) has the \emph{product \sal}
\( \cal F = \prod_\alpha \cal F_\alpha = \bigotimes_\alpha \cal F_\alpha \) defined as
\begin{equation}
	\cal F  = \sigma \ob{ \pi_\alpha^{-1}(E_\alpha) \st \alpha \in A, \ E_\alpha \in \cal F_\alpha},
\end{equation}
i.e.\ \emph{generated by one-dimensional measurable cylinders}.
\asteriskline

If \( A \) is countable, then \( \cal F \) is \emph{generated by cuboids} \( \prod_\alpha E_\alpha \), \( \alpha \in A \) (see~\cite[p.\ 23]{fo}). To prove it, define
\begin{equation}
	\cal F_1 = \sigma \ob{\pi_\alpha^{-1}(E_\alpha) \st \alpha \in A, \ E_\alpha \in \cal F_\alpha}, \quad
	\cal F_2 = \sigma \ob{\prod_{\alpha \in A} E_\alpha \st E_\alpha \in \cal F_\alpha}.
\end{equation}
Firstly, \( \pi_\alpha^{-1}(E_\alpha) = \prod_\alpha E_\alpha \) where \( E_\beta = S_\beta \) for \( \beta \neq \alpha \), so \( \cal F_1 \pd \cal F_2 \).

Secondly,
\begin{equation}
	\prod_{\alpha \in A} E_\alpha = \bigcap_{\alpha \in A} \pi_\alpha^{-1}(E_\alpha) \in \cal F_1,
\end{equation}
so that \( \cal F_2 \pd \cal F_1 \).

\asteriskline

Instead of taking \( E_\alpha \)'s from the \sal s, we can restrict ourselves to \emph{sets of generators}. Suppose \( \sigma(\cal E_\alpha) = \cal F_\alpha \), then (the second part again if \( A \) is countable, and proven the same way)
\begin{equation}
	\bigotimes_{\alpha \in A} \cal F_\alpha = \sigma
	\ob{\pi_\alpha^{-1}(E_\alpha) \st \alpha \in A, \ E_\alpha \in \cal E_\alpha} =
	\sigma \ob{\prod_{\alpha \in A} E_\alpha \st E_\alpha \in \cal E_\alpha}.
\end{equation}

To prove the first part, consider the sets
\begin{equation}
	\cal D = \vit{\pi_\alpha^{-1}(E_\alpha) \st \alpha \in A, \ E_\alpha \in \cal E_\alpha}
	,\quad
	\cal C_\alpha = \vit{
		E \in F_\alpha \st \pi_\alpha^{-1} (E) \in \sigma(\cal D)
	}.
\end{equation}
Then \( C_\alpha \) is a \sal \ containing \( \cal E_\alpha \), which means that
\( C_\alpha = \cal F_\alpha \) and \( \sigma(\cal D) = \bigotimes_\alpha \cal F_\alpha \).

\subsubsection{Borel \sal \ on product metric space}
Suppose \( S_j \) are separable metric spaces (\( j \in \N \)) with Borel \sal s
\( \cal B_j \). Then (see~\cite[p.\ 11]{kalle})
\begin{equation}\label{eq:borelprod}
	\cal B(S_1 \times S_2 \times \cdots) = \cal B_1 \otimes \cal B_2 \otimes \cdots
\end{equation}
Consider the sets
\begin{equation}
	\cal C_1 = \vit{A \st A \text{ open in } \bigtimes_j S_j}, \quad
	\cal C_2 = \vit{S_1 \times \cdots \times B_j \times S_{j+1} \times \cdots \st B_j \in \cal B_j},
\end{equation}
and \( \cal C \) defined similar to \( \cal C_2 \) but where \( B_j \) has to be open instead.
The claim is then \( \sigma(\cal C_1) = \sigma(\cal C_2) \).

Clearly \( \cal C \pd  \cal C_1, \cal C_2 \). But it is also clear that \( \sigma(\cal C) \nd \sigma(\cal C_2) \), so
\( \sigma(\cal C) = \sigma(\cal C_2) \). This proves that the \( \nd \) part in~\eqref{eq:borelprod} holds
without separability.

If \( S_j \) are separable, \( \cal C \) is a topological basis of \( \bigtimes_j S_j \), so that
\( \sigma(\cal C) \nd \sigma(\cal C_1) \). Then \( \sigma(\cal C) = \sigma(\cal C_1) \) as well.



\subsubsection{Measurability in coordinate functions}
See~\cite[p.\ 15]{kalle}. Suppose that \( (\Omega, \cal A) \) and \( (S_j, \cal S_j) \) are measurable
spaces for \( j \in \N \), denoting \( S = \bigtimes_j S_j \). Let \( f \colon \Omega \to S \) be a function and define
\( f_j \) as its \emph{coordinate functions} \( f_j = \pi_j \circ f \). Then
\begin{quote}
	\( f \) is measurable if and only if all \( f_j \) are measurable.
\end{quote}

The \enquote{only if} part is trivial, because each \( f_j \) is a composition of measurable \( f \) and \( \pi_j \).
For the \enquote{if} part, note that \( f \) satisfies the definition of measurability on a generating subset, those being
measurable cuboids:
\begin{equation}\label{eq:coordfunc}
	f^{-1}\ob{B} = \bigcap_j f_j^{-1}(B_j), \quad B = B_1 \times B_2 \times \cdots,
\end{equation}
where \( B \) and all \( B_j \) are measurable sets. \bigskip

\emph{Note} that~\eqref{eq:coordfunc} indeed
holds specifically for cuboid sets \( B \) and not generally. Instead of these cuboids, we could have also
chosen the simpler generating set of one-dimensional cylinders \( \pi_j^{-1}(B_j) \).

\subsection{Convergence theorems}
We consider whether \( \int f_n \to \int f \) if \( f \to f_n \) pointwise, where the functions are defined on a \( \sigma \)-finite measure space \( X \). Following~\cite[\textsection 1.4]{du}, we present
the convergence theorem in order: bounded, Fatou's lemma, monotone, dominated.
Note that in all cases pointwise convergence can be replaced with a.e.\ convergence,
which is weaker but without affecting integration.

\subsubsection{Bounded convergence theorem}
Suppose the \( f_n \) have
\begin{itemize}
	\item \emph{bounded domain}: \( f_n \) vanishes on \( E^c \) for some \( E \) with \( \mu(E)<\infty \),
	\item \emph{bounded range}: \( \abs{f_n} \le M \) uniformly,
	\item \( f_n \to f \) \emph{in measure}.
\end{itemize}
Then \( \int f_n \to \int f \).

To \emph{prove} it, we have:
\begin{align}
	\abs{\int f - f_n} & \le \int \abs{f-f_n}                                             \\
	                   & = \int\limits_{\vit{\abs{f-f_n} \ge \varepsilon}} \abs{f-f_n}
	+\int\limits_{\vit{\abs{f-f_n}<\varepsilon}} \abs{f-f_n}                              \\
	                   & \le 2M \mu \ob{\abs{f-f_n} \ge \varepsilon} + \varepsilon \mu(E)
	\to 0, \quad n \ub, \ \varepsilon \downarrow 0.
\end{align}

We implicitly used that \( \abs f \le M \) as well. We obtain this because \( \abs f \ge M + \varepsilon \)
implies \( \abs {f-f_n} \ge \varepsilon \), giving us \( \mu\ob{\abs f > M} = 0 \) after some simple work.

\subsubsection{Fatou's lemma}
Suppose \( f_n \ge 0 \), then
\begin{equation}
	\liminf_{n \ub} \int f_n \ge \int \liminf_{n \ub} f_n.
\end{equation}

\emph{Recall} that
\begin{equation}
	\liminf_{n\ub} f_n = \sup_m \inf_{n\ge m} f_n,
\end{equation}
thus we define \( g_n = \inf_{m \ge n} f_m  \) so that
\( g_n \uparrow g \coloneq \liminf_{n} f_n \).

It then suffices to show (note \( f_n \ge g_n \))
\begin{equation}
	\liminf_{n\ub} \int g_n \ge \int g.
\end{equation}

Let \( E_m \uparrow X \) be measurable sets with \( \mu(E_m) < \infty \). For fixed \( m \):
\begin{equation}
	(g_n \wedge m) \cdot 1_{E_m} \rightarrow (g \wedge m) \cdot 1_{E_m}.
\end{equation}
Finally,
\begin{equation}
	\liminf_{n\ub} \int g_n \ge
	\int_{E_m} g_n \wedge m \rightarrow
	\int_{E_m} g \wedge m \rightarrow \int g.
\end{equation}
The inequality holds on a subsequence, the first convergence (\( n \ub \)) is the bounded
convergence theorem, and the second (\( m \ub \)) is the below lemma.

\asteriskline

In the proof we used the following \emph{lemma} (see~\cite[p.\ 22]{du}): let \( E_n \uparrow X \)
be measurable sets with \( \mu(E_n) < \infty \). Then
\begin{equation}
	\int_{E_n} f \wedge n \rightarrow \int f.
\end{equation}

Clearly the left side is increasing and below \( \int f \). By definition of the
Lebesgue integral, it suffices to show that for every simple nonnegative \( \varphi \),
there exists \( n \) with
\begin{equation}
	\int \varphi \le \int _{E_n} f \wedge n \le \int f.
\end{equation}
Let \( N \) be such that \( \mu(\varphi > N) = 0 \)
and choose \( n \ge N \). Then
\begin{equation}
	\begin{aligned}
		\int_{E_n} f \wedge n & \ge
		\int_{E_n} \varphi = \int \varphi - \int_{E_n ^c} \varphi \\
		                      & \ge \int \varphi - N\mu(E_n^c),
	\end{aligned}
\end{equation}
so that
\begin{equation}
	\liminf_{n \ub} \int_{E_n} f \wedge n \ge \int \varphi,
\end{equation}
since \( \mu(E_n^c) \un \).

\subsubsection{Monotone convergence theorem}
Suppose \( f_n \ge 0 \) and \( f_n \uparrow f \), then
\begin{equation}\label{eq:monconv}
	\int f_n \rightarrow \int f.
\end{equation}

Since \( f_n \le f \) we have \( \limsup_n \int f_n \le \int f \), and
by Fatou's lemma we have \( \liminf_n \int f_n \ge \int f \).

\subsubsection{Dominated convergence theorem}
Let \( f_n \) be such that \( \abs{f_n} \le g \) uniformly,
for some integrable \( g \). If \( f_n \rightarrow f \), then
\begin{equation}
	\int f_n \rightarrow \int f.
\end{equation}

The condition \( \abs{f_n} \le g \) can be written as \( g \pm f_n \ge 0 \). With \( g \pm f_n \to f \),
we apply Fatou's lemma twice. The \( + \) and \( - \) parts respectively give
\( \int f \ge \limsup_n \int f_n \) and
\( \int f \le \liminf_n \int f_n \).


\subsection{Measure extension}
Carath\' eodory's theorem (multiple variations) extends measure from family of subsets to whole \sal \  (unique?); Kolmogorov's theorem extends measures on finite-dimensional subspaces to a single measure on an infinite-dimensional metric probability space.


\subsection{Product measure}
Suppose \( (S, \cal S, \mu) \) and \( (T, \cal T, \nu) \) are
\( \sigma \)-finite measure spaces. We wish to introduce a measure
on the product measurable space \( (S \times T, \cal S \otimes \cal T) \).

\subsubsection{Sections}
For any \emph{measurable set} \( E \in \cal S \otimes \cal T \), \( s \in S \) and
\( t \in T \) we define the sections
\begin{equation}
	E_s = \vit{t \in T \st (s, t) \in E}, \quad
	E^t = \vit{s \in S \st (s, t) \in E}.
\end{equation}

For any measurable \emph{function} \( f \colon S \times T \to U  \) and
any \( s \in S \) and \( t \in T \) we define the sections
\( f_s \colon T \to U \) and \( f^t \colon S \to U \) with
\begin{equation}
	f_s(t) = f(s,t), \quad f^t(s) = f(s, t).
\end{equation}

We claim that the sections are \emph{always measurable} (sets or functions).
Consider first the set \( E_s \).
Define
\begin{equation}
	\cal D = \vit{E \in \cal S \otimes \cal T \st E_s \in \cal T \text{ for all } s \in S}.
\end{equation}
Clearly, \( \cal D \) is a \( \lambda \)-system and
contains the generating \( \pi \)-system of rectangles
\( A \times B \) since \( (A \times B)_s = B \) if \( s \in A \) and
\( \emptyset \) otherwise. By Dynkin's lemma, \( \cal D = \cal S \otimes \cal T \).

To prove the measurability of functions \( f_s \), it is simple to show that
\( f_s^{-1}(B) = \ob{f^{-1}(B)}_s \in \cal T \), for any measurable \( B \).

\asteriskline
Let \( f \colon S \times T \to \R_{\ge 0} \) be a measurable functions.
We claim that the \emph{maps}
\begin{equation}
	s \mapsto \int_T f(s,t) \ime \nu t, \quad
	t \mapsto \int_S f(s,t) \ime \mu s
\end{equation}
are measurable.

We use a type of Lebesgue's induction.
Assume first that \( \mu \) and \( \nu \) are finite
and that \( f = 1_E \) where \( E = A \times B \).
Then
\begin{equation}
	\int_T 1_E(s,t) \ime \nu t = \nu(E_s) = \nu(B) \cdot  1_A(s).
\end{equation}
Clearly, \( s \mapsto \nu(B)\cdot 1_A(s) \) is measurable. We wish to
extend this to \emph{arbitrary indicator function} so we define
\begin{equation}\label{eq:calDsec3}
	\cal D = \vit{E \in \cal S \otimes \cal T \st s \mapsto \nu(E_s) \text{ is measurable}}
\end{equation}
Again, \( \cal D \) is a \( \lambda \)-system containing
the generating \( \pi \)-system of measurable rectangles, so
\( \cal D = \cal S \otimes \cal T \).

This is then extended to simple functions (linearity of integration)
and to general nonnegative measurable functions. For the latter,
note that monotone convergence of integrals \( \int_T f_n(s,t) \ime \nu t\)
means the pointwise convergence of the maps \( s \mapsto \int_T f_n(s,t) \ime \nu t \),
and the class of measurable functions is closed under
pointwise limits (see~\cite[ex.\ 1.2.7]{du}) \bigskip

\emph{Note.} The requirement \( f \ge 0 \) serves to insure
that the integrals exist. An alternative requirement is
\( \int \abs f \D \xi < \infty \) (with \( \xi \) product measure).

\emph{Note.} The assumption of finite measures (important for \( \cal D \) being a \( \lambda \)-system) does not lose on
generality because of the following: take \( F_n \uparrow T \) with
\( \nu(F_n) < \infty \). Then the maps
\( s \mapsto \nu(E_s \cap F_n) \) converge pointwise to
\( s \mapsto \nu(E_s) \) with pointwise convergence preserving measurability.
The family
\begin{equation}
	\cal D' = \vit{E \in \cal S \otimes \cal T \st
		s \mapsto \nu(E_s \cap F_n) \text{ is measurable for all } n}
\end{equation}
is certainly a \( \lambda \)-system, ensuring the measurability of all
\( s \mapsto \nu(E_s \cap F_n) \).

\subsubsection{Definitions}
We can define a measure \( \xi \) on measurable rectangles with
\begin{equation}\label{eq:defprodm1}
	\xi(A \times B) = \mu(A)\nu(B), \quad A \in \cal S, \ B \in \cal T.
\end{equation}
The function \( \xi \) is then extended to a unique measure on
\( \cal S \otimes \cal T \) by Carath\' eodory-type theorems.

\asteriskline

\emph{Alternatively}, we can define
\begin{equation}\label{eq:defprodm2}
	\xi'(E) = \int_T \mu(E^t) \ime \nu t =
	\int_S \nu(E_s) \ime \mu s,
	\quad E \in \cal S \otimes \cal T.
\end{equation}

\asteriskline

Let us \emph{prove} these definitions are equivalent. Formally using
Fubini's theorem:
\begin{equation}\label{eq:prodmfubini0}
	\begin{aligned}
		\xi(E) & = \int_{S \times T} 1_{E}(u) \ime \xi u           \\
		       & = \int_S \ime \mu s \int_T 1_{E}(s, t) \ime \nu t \\
		       & = \int_S \nu(E_s) \ime \mu s = \xi'(E).
	\end{aligned}
\end{equation}

Conversely, for \( E = A \times B \),
\begin{equation}\label{eq:prodmalt}
	\xi'(E) = \int_S \nu(E_s) \ime \mu s =
	\int_T \nu(B) 1_A(s) \ime \mu s = \mu(A) \nu(B) = \xi(E).
\end{equation}

\asteriskline

We should also \emph{justify} the second equality in~\eqref{eq:defprodm2}.
Define \( \xi'(E) = \int_T \mu(E^t) \ime \nu s \)
and \( \xi''(E) = \int_S \nu(E_s) \ime \mu t \).
By~\eqref{eq:prodmalt} \( \xi' \) and
\( \xi'' \) agree on the generating \( \pi \)-system of
measurable rectangles. By a well known lemma, \( \xi' = \xi '' \).
This argument also suffices to prove \( \xi=\xi' \) (Fubini's theorem is not necessary, \eqref{eq:prodmfubini0} serves a heuristic purpose).

\subsubsection{Fubini's theorem}
Let \( f \colon S \times T \to \R \) be a measurable function such that
\( \int_{S \times T} \abs f \D \xi < \infty \). Then
\begin{equation}\label{eq:prodmfubini}
	\int_{S \times T} f \D \xi
	= \int_S \ime \mu s \int_T f(s, t) \ime \nu t
	= \int_T \ime \nu t \int_S f(s, t) \ime \nu s.
\end{equation}

Consider first a measurable function \( f \ge 0 \).
By previous discussion, the integrals are well-defined and \( \xi \) satisfies:
\begin{equation}\label{eq:prodmfubini2}
	\xi(E) = \int_S \ime \mu s \int_T 1_E(s, t) \ime \nu t, \quad E \in \cal S \otimes \cal T,
\end{equation}
where we also know the order of integration can be reversed. Thus~\eqref{eq:prodmfubini} holds for indicator functions, and is extended to arbitrary
nonnegative functions by linearity and monotone convergence. For general integrable \( f \), apply the previous case to \( f^+ \) and \( f^- \) and subtract. Integrability ensures
that the integrands are measurable and the difference is well-defined.
%\cite[p.\ 25]{kalle} notes that \( f \) can be infinite only on set of measure \( 0 \).; why?

\subsection{Notable exercises}
\subsubsection{Equality of measures agreeing on a generating \( \pi \)-system}
Let \( \mu \) and \( \nu \) be finite measures on measurable space
\( (S, \cal S) \). If
\begin{equation}
	\mu(C) = \nu(C), \quad C \in \cal C,
\end{equation}
where \( \cal C \) is a generating \( \pi \)-system, then \( \mu=\nu \). This can be
extended to \( \sigma \)-finite measures.

To \emph{prove} this, note that
\begin{equation}
	\cal D = \vit{E \in \cal S \st \mu(E) = \nu(E)}
\end{equation}
is a \( \lambda \)-system containing \( \cal C \).

For \emph{\( \sigma \)-finite} measures, note that the previous implies
that \( \mu=\nu \)  when restricted to finite measure spaces
\( (S\cap E_n, \cal S \cap E_n) \), where \( E_n \uparrow S \)
and \( \mu(E_n) < \infty \).
Then for any \( E \in \cal S \)
\begin{equation}
	\mu(E) = \lim_{n \ub} \mu(E \cap E_n) = \lim_{n \ub} \nu(E \cap E_n) = \nu(E).
\end{equation}

\subsubsection{Pointwise convergence preserves measurability}
See~\cite[ex.\ 1.3.7]{du}. Suppose \( f_n \to f \) pointwise with
\( f_n \) measurable. Then \( f \) is measurable because
\begin{equation}
	\vit{f \le a} =
	\bigcup_{r \in \N}
	\bigcup_{k=1}^\infty
	\bigcap_{n \ge k}
	\vit{f_n \le a + \frac 1r}.
\end{equation}

\subsubsection{Tail-formula for \( p \)-th moment}
If \( f \) is a measurable function, then
\begin{equation}
	\norm f_p
	= \ob{p\int_\R y^{p-1} \mu ({\abs {f(x)} \ge y) \D y} }^{1/p}.
\end{equation}

It can be derived, using Fubini for nonnegative functions:
\begin{equation}
	\begin{aligned}
		\norm f_p^p & = \int_\R \abs{f(x)}^p \D x                                    \\
		            & = \int_\R \int_0^{\abs{f(x)}} py^{p-1} \D y \D x               \\
		            & = \int_\R py^{p-1} \int_\R 1_{\cc{0,\abs{f(x)}}}(y)  \D x \D y \\
		            & = p\int_\R y^{p-1} \mu (\abs{f(x)} \ge y) \D y.
	\end{aligned}
\end{equation}

\asteriskline
This holds for all \( 0 < p < \infty \). In probability, this is written
\begin{equation}
	\E(X ^p) = p \int_\R x^{p-1}\P(X \ge x) \D x, \quad X \ge 0.
\end{equation}


\subsubsection{\( \infty \)-norm as limit of \( p \)-norms}
We claim that
\begin{equation}
	\norm f_\infty = \lim_{p \ub} \norm f_p.
\end{equation}
where \( f \) is a measurable function on a probability space.

Let \( \varphi = \sum_{j=1}^n a_j 1_{A_j} \) be a \emph{simple} function
on an \emph{arbitrary} measure space, so that
\( \norm \varphi_p^p = \sum_j \abs {a_j}^p \mu(A_j) \).
Define also \( M = \max \vit{\abs{a_j} \st 1 \le j \le n } = \norm \varphi_\infty  \)
and let \( m \) be the corresponding index.
Clearly \( \norm \varphi_p \le \norm \varphi_\infty \), but also
\begin{equation}
	\norm \varphi_p \ge M \ob{\mu(A_m)}^{1/p} \rightarrow M, \quad p \ub,
\end{equation}
proving the claim for simple functions on arbitrary measure spaces.

Similar ideas are used for general \( f \) on a \emph{probability} space.
Again \( \norm f_p \le \norm f_\infty \). Take arbitrary \( \varepsilon > 0 \)
and set \( A_\varepsilon = \vit{\abs f > M - \varepsilon} \). By definition of
\( M \), \( 0 < \mu(A_\varepsilon) < \infty \). Then,
\begin{equation}
	\norm f_p^p \ge \int_{A_\varepsilon} \abs f^p \D \mu
	\ge (M-\varepsilon)^p \mu(A_\varepsilon).
\end{equation}
Again \( \liminf_p \norm f_p \ge M-\varepsilon \), and the claim follows
by \( \varepsilon \downarrow 0 \).

% \subsubsection{Density of simple functions in \( \L^p \) spaces}

% \subsubsection{Density of step functions in \( \L^p(\R) \) spaces}

%todo:
% luzin, egorov, etc.?



%%%%%%%%%%%%%%


\newpage
\section{Basics of probability}

% uniform integrability
% borel-cantelli
% inequalities
%
%
\subsection{Inequalities}
\subsubsection{Jensen}
Suppose \( X \) is a random variable and \( \varphi \colon \R \to \R \) is
a \emph{convex} function. Then
\begin{equation}
	\varphi(\E X) \le \E (\varphi X).
\end{equation}

For \( c = \E X \), consider an affine function \( h(x) = ax + b \) lying under \( \varphi \) (\( h \le \varphi \))
but with \( h(c) = \varphi(c) \) (the graph of \( h \) is separating two disjoint convex sets: epigraph of \( \varphi  \) and \( \oo{c,\infty} \times \oo{-\infty, c} \)). Then
\begin{equation}
	\varphi(\E X) = h(\E X) = a\E X + b = \E (aX + b) = \E h(X) \le \E \varphi(X).
\end{equation}


\subsubsection{\holder}
For any measurable functions \( f,g \) and real \( p,q,r > 0 \) such that \( 1/r = 1/p + 1/q \),
\begin{equation}
	\norm{fg}_r \le \norm f_p \norm g_q.
\end{equation}

Clearly we can take \( r=1, \norm f_p = 1, \norm g_q = 1 \) (\emph{unless} \( f=0 \) or \( g=0 \) which is a trivial edge-case). It is true that
\begin{equation}\label{eq:holderlemma}
	uv \le \frac{u^p}p + \frac{v^q}q, \quad u,v\ge 0.
\end{equation}

By setting \( u=\abs{f(x)} \) and \( v=\abs{g(x)} \) and integrating, we get
\begin{equation}
	\int \abs{fg} \le p^{-1} \int \abs f^p + q^{-1} \int \abs g^q = p^{-1} + q^{-1} = 1.
\end{equation}

\asteriskline
\emph{Note} that \( 1/p+1/q=1 \) can be written as
\begin{equation}\label{eq:conjuequiv}
	(p-1)(q-1) = 1
\end{equation}
or in the more general case \( (p/r-1)(q/r-1)=1 \) or \( (p-r)(q-r)=r^2 \).

\asteriskline
The inequality~\eqref{eq:holderlemma} can be proven via simple calculus or from
(see~\cite[p.\ 26]{kalle}):
\begin{equation}
	uv \le \int _0^u x^{p-1} \D x + \int_0^v y^{q-1} \D y = \frac {u^p}p + \frac {v^q}q.
\end{equation}
This is because \( uv \) is the area of the rectangle with sides \(  u \) and \( v \)
on \( x \)- and \( y \)- axes respectively, \( \int_0^u x^{p-1} \D x \) is the area
under the graph of \( x^{p-1} \) up to \( u \) and
\( \int_0^v y^{q-1} \D y \) is the area to the left of said graph up to \( v \).
These regions always cover the rectangle with equality when \( x^{p-1} \) passes
through \( (u,v) \), that is \( v = u^{p-1} \).


\subsubsection{Minkowski}
For \( p > 1 \) (the case \( p=\infty \) is easy),
\begin{equation}
	\norm {f+g}_p \le \norm f_p + \norm g_p.
\end{equation}

Firstly,
\begin{equation}\label{eq:prmink1}
	\int \abs{f+g}^p = \int \abs{f+g} \abs{f+g}^{p-1} \le
	\int \abs f \abs{f+g}^{p-1} + \int \abs g \abs{f+g}^{p-1}.
\end{equation}
The integral \( \int \abs f \abs {f+g}^{p-1} \) can be interpreted as
the \( 1 \)-norm of a product, so applying \holder \ we get
\begin{equation}
	\int \abs f \abs{f+g}^{p-1} \le \norm f_p \ob { \norm{f+g}_p^p }^{1/q}.
\end{equation}
With the analogous inequality for the other part of~\eqref{eq:prmink1}
we get
\begin{equation}
	\int \abs{f+g}^p \le \norm{f+g}_p^{p/q} \ug{\norm f_p + \norm g_p}.
\end{equation}
Dividing we get the required result (\( p-p/q =1\)).

\subsubsection{Markov/Chebishev-type inequalities}
The variant is from~\cite[p.\ 29]{du}. For a function \( \varphi \colon \R \to \R_{\ge 0} \),
\( A \) measurable and denoting \( m_A = \inf \vit{\varphi(y) \st y \in A} \),
\begin{equation}
	m_A \P(X \in A) \le \E(\varphi(X) \whi X \in A) \le \E \varphi(X).
\end{equation}
The \emph{proof} is from taking expectations in
\begin{equation}
	m_A 1_A(X) \le \varphi(X) 1_A(X) \le \varphi(X).
\end{equation}

\asteriskline

For \( X \ge 0  \), \( A = \co{a, \infty} \) for some \( a \ge 0 \) and \( \varphi \equiv 1 \) we
get Markov's inequality
\begin{equation}
	\P(X \ge a ) \le \frac{\E X} a.
\end{equation}
If \( X \) is not nonnegative, we need to take \( \varphi(x) = \abs x \).

For \( A = \co{a, \infty} \) for some \( a \ge 0 \) and \( \varphi(x) = (x-\E X)^2 \)
we get Chebishev's inequality
\begin{equation}
	\P( \abs{X - \E X} \ge a  ) \le \frac {\var X}{a^2}.
\end{equation}
The variant with \( \varphi(x) = x^2 \) is also legitimate:
\begin{equation}
	\P(\abs X \ge a) \le \frac{\E X^2}{a^2}.
\end{equation}

\subsubsection{Tails of the normal distribution}
The following is direct (Nourdin, p.\ 39):
\begin{align}
	\ob{\frac 1x - \frac 1{x^3}} e^{-x^2/2}
	 & = \int_x^\infty e^{-y^2/2} \ob{1- \frac 3{y^4}} \D y \\
	 & \le \int_x^\infty e^{-y^2/2} \D y \le
	\frac 1x \int_x^\infty ye^{-y^2/2} \D y = \frac 1x e^{-x^2/2}.
\end{align}

\subsection{Borel--Cantelli lemma}
For events \( A_j \),
\begin{equation}
	\sum_{j \ge 1} \P(A_j) < \infty \implies \P(A_j \io) = 0.
\end{equation}

\emph{Note} that
\begin{align}
	\left\{ A_j \io \right\}  & = \left\{ \sum_{j \ge 1} 1_{A_j} = \infty \right\},   \\
	\left\{ A_j \ult \right\} & = \left\{ \sum_{j \ge 1} 1_{A_j^c} < \infty \right\}.
\end{align}

If \( \sum_j \P(A_j) < \infty \) then
\begin{equation}
	\sum_{j \ge 1} P(A_j) = \sum_{j \ge 1} \E \ob{1_{A_j}} = \E \ob{\sum_{j \ge 1} 1_{A_j}}
\end{equation}
is \( < \infty \), so in particular \( \P\ob{\sum_j 1_{A_j} = \infty }= 0.\)


\asteriskline
The \emph{converse} holds when \( A_j \) are \emph{independent}.
Note
\begin{equation}
	\vit{A_j \io}^c = \ob{\bigcap_{n \ge 1} \bigcup_{k \ge n} A_k}^c
	= \bigcup_{n \ge 1} \bigcap_{k \ge n} A_k^c.
\end{equation}
Then
\begin{equation}
	\begin{aligned}
		1 & = \P(\text{not } A_j \io)                            \\
		  & = \lim_{n \ub} \P\ob{\bigcap_{k \ge n} A_k^c}
		= \lim_{n \ub} \prod_{k \ge n} \ug{1-\P(A_k)}            \\
		  & \le \lim_{n \ub} \exp \ob{- \sum_{k \ge n} \P(A_k)},
	\end{aligned}
\end{equation}
therefore \( \lim_{n \ub} \sum_{k \ge n} \P(A_k) = 0 \) which is equivalent
to \( \sum_{n \ge 1} \P(A_n) < \infty \).

\emph{Note} also that we used \( 1-x \le e^{-x}  \) for all \( x \).

\subsection{Convergence of random variables}
\subsubsection{Characterizing weak convergence I (expectation version)}
It holds that \( X_n \konvd X \), meaning
\( F_n(x) \rightarrow F(x) \) for all \( x \) such that \( F \) is continuous at \( x \),
\emph{if and only if} \( \E g(X_n) \to \E g(X) \) for all
\emph{continuous and bounded} functions \( g \).

First \emph{assume weak convergence}. There exist variables \( Y_n, Y \) with distributions
\( F_n, F \) such that \( Y_n \konvas Y \) (see below). Then also \( g(Y_n) \konvas g(Y) \) and
\begin{equation}
	\E g(X_n) = \E g(Y_n) \rightarrow \E g(Y) = \E g(X)
\end{equation}
using the \emph{bounded convergence} theorem.

Now the \emph{converse}. We want \( \E g(X_n) \to \E g(X) \)
where \( g = 1_{ \oc{\infty, x}  } \), but these functions are not continuous. The idea
then is to approximate them by continuous \( g \)'s.

Fix an \( x \in \R \) and for all \( r > 0 \) define
\begin{itemize}
	\item \( g_{r+} \) as \( 1 \) up to \( x \), \( 0 \) from \( x+1/r \) onwards and linear inbetween;
	\item \( g_{r-} \) as \( 1 \) up to \( x - 1/r \), \( 0 \) from \( x \) onwards and linear inbetween.
\end{itemize}
Then we have
\begin{equation}
	\limsup_{n \ub} \P(X_n \le x) \le \lim_{n \ub} \E g_{r+}(X_n) = \E g_{r+}(X) \to \P(X \le x).
\end{equation}
The lattermost convergence holds for all \( x \) due to
\begin{equation}
	\P(X \le x) \le \E g_{r+} (X) \le P(X \le x + 1/r)
\end{equation}
and the right-continuity of \( F \).

The \( \liminf \) side is obtained using \( g_{r-} \):
\begin{equation}
	\liminf_{n \ub} \P(X_n \le x) \ge \lim_{n \ub} \E g_{r-}(X_n) = \E g_{r-}(X) \to \P(X \le x),
\end{equation}
with the lattermost convergence being true for \( x \not \in \discont F \), since only then does
\begin{equation}
	\P(X \le x - 1/r) \le \E g_{r-}(X) \le \P(X \le x)
\end{equation}
yield the desired sandwiching.

\asteriskline
Earlier we used the following \emph{lemma}: if \( F_n \konvw F \) there exist
variables with those distributions such that \( Y_n \konvas Y_n \).
The common probability space is defined as usual
\begin{equation}
	\Omega = \cc{0,1}, \quad \mathcal F = \text{Borel sets}, \quad \P = \text{Lebesgue measure on} \cc{0,1},
\end{equation}
and the variables by
\begin{equation}
	Y_n(\omega) = \sup \vit{x \in \R \st F_n(x) < \omega}, \quad \omega \in \cc{0,1}.
\end{equation}

For proof that \( Y_n \konvas Y \) see~\cite[p.\ 118]{du}.

\subsubsection{Characterizing weak convergence II (portmanteau theorem)}


\subsubsection{Characterizing convergence in probability}

\subsubsection{Continuous mapping theorems}
Assume \( X_n \konvas X \) with \( g \) a measurable function such that
\( \P(X \in \discont g) = 0 \). \emph{Then} \( g(X_n) \konvas g(X) \).
This simply follows from
\begin{equation}
	\vit{ g(X_n) \to g(X)} \nd \vit{X_n \to X} \cap \vit{X \not \in \discont g}
\end{equation}

\asteriskline

Assume \( X_n \konvd X \) with \( g \) a measurable function such that
\( \P(X \in \discont g) = 0 \). \emph{Then} \( g(X_n) \konvd g(X) \). In particular and
if \( g \) is bounded, \( \E g(X_n) \to \E g(X) \).

Again let \( Y_n \konvas Y \). For any bounded continous \( f \) is
\( f \circ g \) also bounded and \( \discont(f \circ g) \pd \discont g \).
Thus \( f(g(Y_n)) \konvas f(g(Y)) \) and \( \E f(g(Y_n)) \to \E f(g(Y)) \) follows
by the bounded convergence theorem.

\asteriskline

Assume \( X_n \konvp X \) with \( g \) a measurable function such that
\( \P(X \in \discont g) = 0 \). \emph{Then} \( g(X_n) \konvp g(X) \).

See~\cite[p.\ 103]{kalle}.
Fix a subsequence \( N' \pd \N \).
The convergence \( X_n \konvp X \)
implies the convergence \( X_n \konvas X \) along some \( N'' \pd N' \).
Then also \( g(X_n) \konvas g(X) \) along \( N'' \) and
\( g(X_n) \konvas g(X) \) due to the uniqueness of the a.s.-limit. Since \( N' \) was arbitrary,
it follows that \( g(X_n) \konvp g(X) \).

\subsubsection{a.s. implies \( \P \)}
For any \( \varepsilon > 0 \),
\begin{align}
	\vit{X_n \not \to X} & \nd \vit{ \abs{X_n - X} \ge \varepsilon \text{ i.o.}}                   \\
	                     & \implies \P \ob{\abs{X_n-X} \ge \varepsilon \text{ i.o.}} = 0           \\
	                     & \implies \sum_{n = 1}^\infty \P(\abs{X_n - X} \ge \varepsilon) < \infty \\
	                     & \implies \lim_{n \ub} \P(\abs{X_n-X} \ge \varepsilon) = 0.
\end{align}

\subsubsection{\( \L^p \) implies \( \P \)}
We use a Markov-type inequality
\begin{equation}
	\P(\abs{X_n - X} \ge \varepsilon) \le \frac{\E \abs{X_n-X}^p}{\varepsilon^p} \un.
\end{equation}
\asteriskline

The \emph{converse} will hold if \( X_n-X \) is bounded by some \( M > 0 \), which is specifically true
if both \( X \) and all \( X_n \) are bounded. Then
\begin{equation}
	\E \abs{X_n-X}^p = \E \ob{\abs {X_n-X}^p \wedge M^p } \le
	\varepsilon \P(\abs{X_n-X} < \varepsilon) +
	M^p \P(\abs{X_n-X} \ge \varepsilon)
\end{equation}
which goes to \( 0 \) as \( \varepsilon \downarrow 0 \).

\subsubsection{\( \P \) implies d}
\subsubsection{Cauchyness in probability and in \( \L^p \)}


\subsection{Uniform integrability}
\subsubsection{Definition and characterization}
See~\cite[p.\ 106]{kalle}. A family \( (X_j)_{j \in J} \) of random variables is uniformly integrable if
\begin{equation}
	\lim_{r\ub} \sup_{j \in J} \E \ob{\abs{X_j} \whi \abs {X_j} > r} = 0.
\end{equation}
If it's a sequence \( (X_n)_{n \in \N} \) of \emph{integrable} variables, then
\begin{equation}
	\lim_{r\ub} \limsup_{n \ub} \E \ob{\abs{X_n} \whi \abs {X_n} > r} = 0
\end{equation}
\emph{suffices}, because
\begin{equation}
	\lim_{r \ub} \ug{ \limsup_{n \ub} \E \ob{\abs{X_n} \whi \abs{X_n} > r} - \sup_{n \in \N} \E \ob{\abs{X_n} \whi \abs{X_n} > r} } = 0,
\end{equation}
since both parts themselves tend to \( 0 \).

\asteriskline
For uniform integrability, boundedness in \( \L^p \) for some \( p > 1 \) (meaning \( \sup_j \E \abs{X_j}^p < \infty \)) \emph{suffices}.
We use the Markov-type inequality: \( \P(\abs{X_j} \ge x) \le x^{-p}\E\abs{X_j}^p  \) to get
\begin{align}
	\E (\abs{X_j} \whi \abs{X_j} > r ) & = \int_0^\infty \P(\abs{X_j} \cdot 1_{\vit{\abs{X_j} > r}} \ge x) \D x
	= \int_r^\infty \P(\abs{X_j} \ge x) \D x                                                                    \\ &\le \E \abs{X_j}^p \int_r^\infty x^{-p} \D x
	= \frac 1{p-1}r^{1-p} \E \abs{X_j} ^p.
\end{align}
for all \( j \in J \) and \( r > 0 \), though~\cite{kalle} says something different.

\asteriskline
Now the main \emph{characterization} of uniform integrability: \( (X_j)_j \)
are uniformly integrable if and only if
\begin{equation}
	\sup_{j \in J} \E \abs{X_j} < \infty \quad \text{and} \quad \lim_{\P(A) \un} \sup_{j \in J}
	\E(\abs{X_j} \whi A) = 0.
\end{equation}
\emph{Only if.}
Set arbitrary \( \varepsilon > 0 \) and choose \( r_0 \) so that
\( \sup_j \E(\abs{X_j} \whi \abs{X_j} > r) < \varepsilon \) for \( r \ge r_0 \).
\begin{align}
	\E \abs{X_j} & = \E (\abs{X_j} \whi \abs{X_j} \le r_0) + \E(\abs{X_j} \whi \abs{X_j} > r_0) \\
	             & \le r_0 + \E(\abs{X_j} \whi \abs{X_j} > r_0)
	\le r_0 + \varepsilon
\end{align}
This proves the first part.

More generally for any \( r > 0 \),
\begin{align}
	\E(\abs{X_j} \whi A) & = \E(\abs{X_j} \whi \abs{X_j} \le r, A)
	+ \E(\abs{X_j} \whi \abs{X_j} >r, A)                                     \\
	                     & \le r\P(A) + \E(\abs{X_j} \whi \abs{X_j} > r) \un
\end{align}
as \( \P(A) \un \) and \( r \ub \) and uniformly in \( j \). This proves the second part.

\emph{If.} Define \( A_{j,r} = \vit{\abs{X_j} > r} \). Then the probabilities of these events are bounded
uniformly in \( j \) by Markov's inequality:
\begin{equation}
	\sup_j \P(A_{j,r}) \le r^{-1} \sup_j \E \abs{X_j}.
\end{equation}

For any \( \varepsilon > 0 \) and large enough \( r \), the assumption implies
\begin{equation}
	\sup_{j,k \in J } \E(\abs{X_j} \whi A_{k, r}) < \varepsilon,
\end{equation}
which in particular gives the desired uniform integrability by taking only \( k=r \).

\asteriskline
A similar technique can be used to prove the following: if \( X \) is integrable,
then
\begin{equation}
	\lim_{\P(A) \un} \E (\abs X \whi A) = 0.
\end{equation}
We know that
\begin{equation}
	\lim_{r \ub} \E(\abs X \whi \abs X > r) = 0
\end{equation}
because \( X \cdot 1_{\vit{X > r}} \un \) and dominated convergence. The
\enquote{rest of \( A \)} is not a problem since
\begin{equation}
	\E(\abs X \whi A) \le r\P(A) + \E(\abs X \whi \abs X > r).
\end{equation}

\subsubsection{Convergence of means}
For random variables \( X, X_1, X_2, \ldots \) with \( X_n \konvd X \) we have
\begin{equation}\label{eq:unifco-conmean}
	\E X_n \to \E X < \infty \iff  X_n \text{ are uniformly integrable.}
\end{equation}

\emph{\fbox { \( \Leftarrow \) }}  To get \( \E X < \infty \), we switch to \emph{bounded} functions. For all \( r > 0 \),
\begin{equation}
	\liminf_{n \ub} \E X_n \ge \lim_{n \ub} \E \ob{X_n \wedge r} = \E \ob{X \wedge r}
\end{equation}
so with \( r \ub \) we have \( \E X \le \liminf_n \E X_n < \infty  \).

The other part follows from
\begin{equation}
	\abs{\E X_n - \E X}
	\le \abs{\E X_n - \E \ob {X_n \wedge r} }
	+ \abs{\E \ob{X_n \wedge r} - \E \ob{X \wedge r} }
	+ \abs{\E \ob{X \wedge r} - \E X}
\end{equation}
for all \( r > 0 \), wherein we first let \( n \ub \) and then \( r \ub \), the uniformity being key.

\emph{\fbox { \( \Rightarrow \)}}
If \( X \) is \emph{continuous}, then \( \E(X_n \whi X_n > r) \to \E(X \whi X > r) \) by
the continuous mapping theorem.

More generally, we need to replace \( x \mapsto x \cdot 1(x > r) \) with a \( \ge \) continuous function;
\cite{kalle} offers
\begin{equation}
	g(x) = x - \ug{x \wedge (r-x)_+}
\end{equation}
which is more nicely written as
\( g(x) = 0 \) if \( x \le r/2 \); \( 2x-r \) if \( r/2 < x \le r \) and \( x \) otherwise. Then
\begin{equation}
	\E\ob{X_n \whi X_n > r} \le \E{g(X_n)} \stackrel n \rightarrow \E{g(X)} \stackrel r \rightarrow 0
\end{equation}
with, again, obvious uniformity in \( n \).


% \subsection{Stochastic processes and finite-dimensional distributions}

\subsection{Independence and \zerone \ laws}

% \subsection{Constructing random variables with given distributions}

\subsection{Notable exercises}
\subsubsection{Asymptotics of tail-expectations}
See~\cite[ex.\ 1.6.14]{du}. Let \( X \ge 0 \) (without assuming \( E(1/X) < \infty \)). Then
\begin{equation}
	\lim_{y \ub} y\E \ob{\frac 1X \whi X > y} = 0, \quad
	\lim_{y \downarrow 0} y\E \ob{\frac 1X \whi X > y} = 0.
\end{equation}

For the \emph{first} part,
\begin{equation}
	\E \ob{\frac yX \whi X > y} \le \E (1\whi X > y) = \P(X > y) \un.
\end{equation}

For the \emph{second} part,
denoting \( n = 1/y \) and replacing \( X \) with \( 1/X \), the limit becomes
\( \lim_{n \ub} \frac 1n \E(X; X < n) \) and we have
\begin{equation}
	\frac 1n \E(X \whi X \le n) = \frac 1n \int_0^n x \D F
	= \frac 1n \int_0^{\sqrt n} x \D F + \frac 1n \int_{\sqrt n} ^n x\D F \un,
\end{equation}
because
\begin{equation}
	\int_0^{\sqrt n} x \D F \le \sqrt n, \qquad
	\int_{\sqrt n}^n x \D F \le n \ug{F(n) - F(\sqrt n)}.
\end{equation}

\subsubsection{For \( p < q \), \( \L^q \)-convergence implies \( \L^p \)-convergence}
See~\cite[ex.\ 1.6.11]{du}. Let \( 0 < p < q \) and \( \E \abs X ^q < \infty \). Then
\begin{equation}
	\E \abs X^p \le \ob{\E \abs X^q}^{p/q}.
\end{equation}
In particular,
\begin{itemize}
	\item the \( p \)-norm is \emph{smaller}; by taking the root we get \( \norm X_p \le \norm X_q \),
	\item the \( p \)-norm is \emph{finite}, so if \( X \) is \( q \)-integrable it is \( p \)-integrable,
	\item \emph{convergence} in \( \L^q \) implies convergence in \( \L^p \).
\end{itemize}

To \emph{prove} it, note that
\begin{equation}
	\E \abs X^p = \E \ug{\ob{\abs X^q}^{p/q}}
	\le \ob{\E \abs X^q}^{p/q}
\end{equation}
where the inequality is Jensen for concave \( x \mapsto x^{p/q} \).

\subsubsection{Integrating with respect to Lebesgue--Stieltjes measures}
For \( F, G \) corresponding to \( \mu, \nu \) measures on \( (\R, \mathcal B) \),
\begin{enumerate}[label=(\alph*)]
	\item \( \int_{\oc{a,b}} \ug{F(y)-F(a)} \ime Gy =
	      (\mu \otimes \nu) \vit{(x, y) \st a < x \le y < b} \),
	\item
	      \( \displaystyle \int_{\oc{a,b}} F(y) \ime Gy + \int_{\oc{a,b}} G(y) \ime Fy
	      = F(b)G(b) - F(a)G(a) + \sum_{x \in \oc{a,b} } \mu(\{x\})\nu(\{x\}). \)
	\item In particular, if \( F=G \) is continuous,
	      \begin{equation}
		      \int_{\oc{a,b}} 2F(y) \ime Fy = F(b)^2-F(a)^2.
	      \end{equation}
\end{enumerate}

\emph{Firstly}, denote \( D_{a,b} = \vit{(x,y) \st a < x \le y \le b} \).
Then,
\begin{align}
	(\mu \otimes \nu)(D_{a,b}) & = \int \mu(D_{a,b}^y) \ime \nu y           \\
	                           & = \int_{\oc{a,b}} \mu(\oc{a,y}) \ime \nu y \\ &=
	\int_{\oc{a,b}} \ug{F(b)-F(a)} \ime Gy.
\end{align}

\emph{Secondly},
\begin{align}
	\int_{\oc{a,b}} F(y) \ime Gy & = \int_{\oc{a,b}} \ug{F(y)-F(a) + F(a)} \ime Gy                \\
	                             & = \int_{\oc{a,b}} \ug{F(y)-F(a)} \ime Gy + F(a) \ug{G(b)-G(a)} \\ &=
	(\mu \otimes \nu)(D_{a,b}) + F(a)\ug{G(b)-G(a)}.
\end{align}
Analogously,
\begin{align}
	\int_{\oc{a,b}} G(y) \ime Fy = (\mu \otimes \nu)(D_{a,b}') + G(a) \ug{F(b)-F(a)}
\end{align}
where \( D_{a,b}' = \vit{(x,y) \st a < y \le x \le b} \).
The final result follows from
\begin{align}
	( \mu \otimes \nu)(D_{a,b}) + (\mu \otimes \nu)(D_{a,b}')
	 & = (\mu \otimes \nu)(D_{a,b} \cup D_{a,b}') + (\mu \otimes \nu)(D_{a,b} \cap D_{a,b}') \\
	 & = \ug{F(b)-F(a)}\ug{G(b)-G(a)} + \sum_{x \in \oc{a,b}} \mu(\{x\}) \nu(\{ x\}).
\end{align}



\newpage
\printbibliography

\end{document}
